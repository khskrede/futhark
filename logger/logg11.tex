\documentclass[a4paper, norsk, 12pt]{report}
\usepackage[Glenn]{fncychap}
\usepackage[norsk]{babel}
\usepackage[latin1]{inputenc}
\usepackage{fancyhdr}
\usepackage{palatino}
\usepackage{hyperref}
\hypersetup{colorlinks=true}
\topmargin=-0.95in
\evensidemargin=0in
\oddsidemargin=0in
\textwidth=6.5in
\textheight=10.0in
\headsep=0.25in


\begin{document}

\chapter*{Gruppelogg dag 11, Futhark}

\section*{Oppsummering}

Dagen idag kan oppsummeres som en typisk jobbedag, hvor gruppa har v�rt delt
store deler av dagen. Paul og Joakim behandlet data fra baconfors�k, mens
�smund, Knut og Turid jobbet videre med programmet, endelig ser vi fornuftige
resultater fra datamodellen.

Etter lunsj var det en meget god gruppeaktivitet i regi l�ringsassistentene.
Gruppemedlemmene skulle ved hjelp av poeng, rangere seg selv og gruppemedlemmer
i ulike situasjoner. Vi hadde p� forh�nd bestemt oss for at presentasjonen av
gruppa i prosessrapporten skulle inneholde to aspekter pr. gruppemedlem.
\begin{itemize}
\item Gruppemedlemmets egen oppfatning av seg selv
\item Gruppas oppfatning av personen
\end{itemize}
Alts� var denne biten av prosessrapporten en naturlig forlengelse av �vingen, og
dermed noe vi f�lte vi hadde stort utbytte av.

Verdt � kommentere at gruppa (tilsynelatende) virker meget trygg p� hverandre,
det var ingen pinlige og/eller flaue situasjoner i forbindelse med
poenggivningen, i tillegg til at v�r selvoppfatning s� ut til � passe godt med
gruppas oppfatning som helhet.

Gruppa har likevel merket seg at hver enkelts oversikt over prosjektets totale
fremgang er noe vag, Joakim foreslo derfor at det for hver gang skulle skrives
ned en fremdriftsplan, hvor vi oppsummerte dagens gj�rem�l og sammenlignet med
planlagt aktivitet. I tillegg til at vi bestemte oss for hva som skulle v�re
gjort til neste sammenkomst. Dette var det bred enighet om i gruppa, og man kan
godt si at dette fremprovoserer bruken av flere av Scharwz' grupperegler,
eksempelvis nr. $7$: ``Formuler fremtidige m�l i fellesskap''.

Gruppa f�ler at fremdriften for tiden er god, i tillegg til tryggheten innad er
�kende. Disse egenskapene henger utvilsomt sammen, det at vi som gruppemedlemmer
er trygge p� hverandre f�rer til bedre kommunikasjon og f�rre misforst�elser,
noe som igjen leder til �kt produktivitet og effektivitet.
\end{document}
