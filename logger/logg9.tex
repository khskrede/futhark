\documentclass[a4paper, norsk, 12pt]{report}
\usepackage[Glenn]{fncychap}
\usepackage[norsk]{babel}
\usepackage[utf8]{inputenc}
\usepackage{fancyhdr}
\usepackage{palatino}
\usepackage{hyperref}
\hypersetup{colorlinks=true}
\topmargin=-0.95in
\evensidemargin=0in
\oddsidemargin=0in
\textwidth=6.5in
\textheight=10.0in
\headsep=0.25in


\begin{document}

\chapter*{Gruppelogg dag 9, Futhark}

\section*{Oppsummering}
Gruppen startet dagen med å ha presentasjon av sitt tema. Det gikk greit, folk
var godt forberedte, og vi oppnådde gruppens mål om å ha en kort og konsis
presentasjon. Vi merker også en forbedring fra første gang vi presenterte
gruppens problemstilling, da gruppens medlemmer uttrykte misnøye med å
presentere. Dette skyldes nok både at vi er tryggere på hverandre, men også at
gruppa er tryggere på landsbyen. Vi merket også at vårt forsøk på humor ble
ganske godt mottatt.

Like etter oppstart, under gruppas morgenmøte, ble det stilt spørsmål ved om vi
skal skrive rapportene på engelsk eller norsk. Gruppa tok en beslutning ved at
hvert medlem først presenterte sitt syn, og så ble vi enig om en felles
plattform som passet med gruppa som helhet. Dette harmonerer bra med Johnson \&
Johnson (1994) som foreskriver 4 steg som er effektive for å skape og forbedre
en felles gruppeidentitet: presenter egne meninger, lytt til andres meninger,
dann en felles gruppeidentitet, anta denne identiteten. Det at vi underbevisst
bruker denne strategien som gruppe for å ta avgjørelser kan utvilsomt settes i
sammenheng med at vi har høy effektivitet og høy grad av positiv samhandling i
gruppen.

Dagen har vært en god arbeidsdag. Knut Halvor har en implementasjon av flere av
de likningene som skal løses, og vi får nokså fornuftige foreløpige resultater.
Joakim og Paul brukte mye av dagen på å faktisk steke bacon i en mikrobølgeovn,
med nokså fornuftige resultater. Denne mikrobølgeovnen var desverre litt dårlig,
så vi skaffer en ny til neste gang og satser på bedre resultater da. Åsmund har
spilt flere roller, han hjalp Knut Halvor med å finne passende fysiske
konstanter og effektledd til modellene, og han hjalp Joakim og Paul med å skaffe
utstyr og oppsettet for å gjøre de målingene som trengtes. Turid har vært litt småsyk og
bidratt litt mindre enn vanlig, men hun har stort sett hjulpet Knut Halvor med
de numeriske løsningene.

Alle gruppemedlemmene opplever at deres roller i gruppa passer godt til deres
forutsetninger. Åsmund var spesielt fornøyd med at hans fysikkkunskaper kunne
brukes både til å hjelpe til med det numeriske arbeidet og med det
eksperimentelle. 

Som en gruppe arbeider vi godt, noe som kan forklares med høy teknisk
diversitet, som passer godt med at vi er en gruppe som skal produsere et
resultat. At Turid var lite aktiv i dag hadde ikke en veldig utpreget effekt på
gruppas samspill. Turid er vanligvis en samlende og styrende faktor i gruppa, men 
det har oftest vært viktig når vi skal arbeide med prosess, og det skjedde ikke
så mye prosess i dag, noe vi tror er forklaringen på dette. Gruppemedlemmenes
personlige engasjement for arbeidsoppgaven var det som holdt oss i gang, nok et
kjennetegn på en gruppe som har antatt en felles identitet og der
gruppemedlemmene har engasjement for å oppnå gruppas mål.

\end{document}


