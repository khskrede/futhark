\documentclass[11pt]{article}
\usepackage{graphicx}
\usepackage{url}
\usepackage[utf8]{inputenc} 
\author{Futhark}
\title{Gruppelogg dag 10}
\setlength{\parindent}{0pt}
\setlength{\parskip}{2ex} 
\begin{document}
\maketitle

\section{Oppsummering}

Dagen startet med presentasjon av hvordan prosessrapporten burde settes opp, og
hva den burde inneholde. Deretter ble det gitt i oppgave å omskrive gruppekontrakten.

Gruppen diskuterte da hvilken effekt den daværende gruppekontrakten hadde hatt på arbeidet.
Uten at medlemmene har vert obs på det, ble punktet, "alle utdelte oppgaver skal ha en tidsfrist" totalt ignorert.

Dette skyldes at gruppen ikke har hatt en klar fremdriftsplan, noe den nye gruppekontrakten
reflekterer.

Punktene som ble fjernet fra gruppekontrakten:\\
- Alle oppgaver som deles ut skal føres inn på wiki-en.

Punktene som ble lagt til i gruppekontrakten:\\
- Kommunikasjon skal gjøres tydelig mellom medlemmene i gruppa.\\
- Gruppa skal etablere et fremdriftskart.

I tillegg til dette ble det diskutert hvilket språk prosessrapporten skulle skrives på. Her
brukte vi samme fremgangsmåte som sist, men denne gang var det ikke enstemmig, 
alle medlemmene fikk ytret sin mening og en beslutning ble
tatt om at den skal skrives på norsk. Noe som igjen appelerer Johnson \& Johnson (1994) sine 4 steg
for å skape felles gruppeidentitet.

Vi kom sent igang med prosjektarbeidet, da omskrivingen av gruppekontrakten og diskusjon
rundt dette tok all tid frem til lunsj.

Straks etter lunsj fikk Aasmund og Joakim hentet utstyr til og utføre forsøk.
Dette var essensielt da Joakim og Paul har brukt resten av dagen på og utføre og dokumentere
disse forsøkene.
Knut Halvor har fortsatt jobben med å implementere de numeriske likningene.
Aasmund har fortsatt sitt arbeid med å finne likninger som beskriver de fysiske prosessene
som oppstår. Turid hjelpte deretter til med og diskretisere løsningene slik at de er klare til
at Knut Halvor kan implementere de.

\end{document}


