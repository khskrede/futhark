\documentclass[11pt]{article}
\usepackage{graphicx}
\usepackage{url}
\usepackage[utf8]{inputenc} 
\renewcommand{\baselinestretch}{1.5}
\author{Futhark}
\title{Gruppelogg dag 5}
\begin{document}
\maketitle

Gruppa drifter ganske ofte vekk fra oppgaven og prater om andre ting. Dette irriterer
noen gruppemedlemer som prøver å få igang diskusjonen om oppgaven igjen. Det viser seg
og at det er lett og flytte oppmerksomheten til gruppa etter en slik divergens.

Knut Halvor var skeptisk til og bruke de mest kompliserte og presise ligningene som gruppen
hadde foreslått fordi det kunne resultere i for mange ukjente som måtte approksimeres. Han mente
også at det ville ta for lang tid og bli for komplisert til og gå rett på denne problembeskrivelsen.

En enklere modell ble foreslått og gruppen aksepterte at dette ville være en første prioritet.
Den kompliserte modellen inneholder bygger på den enklere modellen, så dette ble akseptert som et 
godt utgangspunkt og bygge videre.

Iløpet av dagen fikk vi skrevet ferdig en presentasjon av oppgaven, og litt om hvordan vi
skulle løse den. Deadline for presentasjonen var på slutten av dagen, noe som gjorde at gruppen
ble mer fokusert for og få den ferdig. 

På slutten av dagen tok Turid aksjon og delte ut oppgaver gruppemedlemmene skulle se på
før neste møte. Dette var fordi hun var frustrert over manglende initiativ.

Dette viser at gruppen har utviklet seg så langt at gruppemedlemer føler seg trygge nok til og
ta ansvar. 

Presentasjonen på slutten av dagen føltes som en samlende opplevelse da de fleste medlemmene
av gruppen ikke føler seg 100 prosent komfortable i gjennomføringen av en slik presentasjon. Medlemmene
av gruppen hjalp hverandre under presentasjonen når noen stod fast. Dette viser at det er en
god gruppekultur.

\end{document}


