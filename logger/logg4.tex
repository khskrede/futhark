\documentclass[a4paper, norsk, 12pt]{report}
\usepackage[Glenn]{fncychap}
\usepackage[norsk]{babel}
\usepackage[utf8]{inputenc}
\usepackage{fancyhdr}
\usepackage{palatino}
\usepackage{hyperref}
\hypersetup{colorlinks=true}
\topmargin=-0.95in
\evensidemargin=0in
\oddsidemargin=0in
\textwidth=6.5in
\textheight=10.0in
\headsep=0.25in


\begin{document}

\chapter*{Gruppelogg dag 4, Futhark}

\section*{Oppsummering}
Dagen startet litt tregt med mye trøtthet, og kaffe. Etter hvert tok det seg
litt opp, og vi gikk på datarommet for å google mer. Etter omtrent en time med
googling returnerte vi til grupperommet og diskuterte. Vi diskuterte en bedre
formulering av modellen, og fikk etter hvert formulert en problemstilling. Så
tok vi lunsj. Etter lunsj kom brannalarmen og ødela litt, så brukte vi tiden på
å forberede presentasjon neste gang.

\section*{Noterte hendelser}
Matematikerne på gruppa følte et visst press på å bruke elementmetoden, men de
mente selv at differansemetoden kanskje var bedre egnet for gruppa. De følte at dette var
et vanskelig tema å ta opp, så det tok litt tid før man fikk tatt det opp. Men
når det ble tatt opp var alle i gruppa åpne for dette synspunktet, og det ble
enighet om å la matematikerne vurdere det de kan best. Lærdommen vi kan dra av
dette er at man ikke må vente med å ta opp et tema man føler er vanskelig å ta
opp, særlig når temaet ikke berører noen personlig. Det kan fort hende at de
andre ikke har en særlig bestemt holdning, men prøver å finne konsensus.

Joakim og Turid var uenige om stabiliteten til Crank-Nicholson-metoden. Turid
ble veldig fornøyd og konkluderte med ``in your face'' når wikipedia var enig
med henne. Dette skapte litt dårlig stemning, fordi ingen var åpne for den
andres synspunkt, og tok en faglig uoverenstemmelse veldig personlig. Dette kan
ses på som et brudd mot Schwarz' sjette regel for godt gruppesamspill. For å hindre
at dette gjentar seg, må man i fremtiden være mer åpne med sine innspill, og ikke 
ta det så personlig når man tar feil. Men man må også passe på at man ikke
kveler det sunne konkurranseinstinktet som finnes i gruppa, noe som kan gi bedre
resultater enn et mer homogent miljø.

Det noteres at gruppedynamikken har blitt ganske god iløpet av disse ukene, og
debatten rundt problemstilling var frisk og nyansert. Diskusjonene fører til noe
konkret, og folk er flinke til å myldre og å bygge på andres ideer. Når gruppa
er enige om at nå skal vi gjøre noe, tar folk personlig initiativ uten at andre
pusher på. Vi bemerker at det kanskje er et forbedringspotensiale på å spisse
inn kortsiktige målsetninger når vi skal søke etter informasjon, det kan
fort bli ineffektivt når alle gjør veldig brede søk.

\end{document}


