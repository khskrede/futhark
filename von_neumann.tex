\documentclass[a4paper,norsk]{article}
\usepackage[latin1]{inputenc}
\usepackage[T1]{fontenc}
\usepackage{babel, textcomp}
\usepackage{graphicx}
\usepackage{fancyvrb}
\usepackage{amsmath}
\usepackage{multirow}
\usepackage[margin=1.5in]{geometry}
\setlength{\parindent}{0in}

\tolerance = 10000             
\hbadness = \tolerance         
\pretolerance = 2000
\title{�ving 8 \\ TKT4192 Elementmetode i styrkeanalyse}
\author{Joakim Johnsen - joakij@stud.ntnu.no}
\date{\today}

\begin{document}

\section{Von Neumann-analyse av Crank-Nicolson-skjemaet}

\begin{equation}
\frac{\partial u}{\partial t} = \alpha\frac{\partial^2u}{\partial^2x}
\end{equation}

\noindent Ved � anvende Crank-Nicolson oppn�s f�lgende iterasjonsskjema

\begin{equation}
u_i^{n+1} = u_i^n\left(1-2D\right)+D\left(u_{i+1}^{n+1}-2u_i^{n+1}+u_{i-1}^{n+1}\right) + D\left(u_{i+1}^n + u_{i-1}^n\right)
\end{equation}
\begin{equation}
D = \frac{1}{2}\frac{\alpha\Delta t}{(\Delta x)^2}
\end{equation}

\noindent Innf�rer tiln�rmelsen $u_j^n \approx E_j^n$ hvor $E_j^n = G^ne^{i\beta x_j}$, her er $n$ en potens. Innsatt i $(1)$:

\begin{equation*}
G^{n+1}e^{i\beta x_j} = G^ne^{i\beta x_j}\left(1-2D\right) + G^{n+1}D\left(e^{i\beta x_{j+1}} - 2e^{i\beta x_j} + e^{i\beta x_{j-1}}\right) + G^nD\left(e^{i\beta x_{j+1}} + e^{i\beta x_{j-1}}\right)
\end{equation*}

\noindent Hvis vi deler $(3)$ p� $G^ne^{i\beta x_j}$ og benytter at $x_{j+1} \approx x_j + h$ f�r vi
\begin{eqnarray*}
G\left[1-D\left(2\cos{\beta h} - 2\right)\right] &=& 1 - 2D\left(1-\cos{\beta h}\right) \\
\cos{\beta h} &=& \frac{1}{2}\left(1-\sin^2{\frac{\beta h}{2}}\right) \\
G\left(1+4D\sin^{2}{\frac{\beta h}{2}}\right) &=& 1 - 4D\sin^2{\frac{\beta h}{2}} \\
G &=& \frac{1-4D\sin^2{\frac{\beta h}{2}}}{1+4D\sin^2{\frac{\beta h}{2}}}
\end{eqnarray*}

\noindent Vi vet at den maksimale verdi av $\sin^2{x}$ er $1$, og at iterasjonsskjemaet er stabilt dersom $|G| \le 1$, innsatt:

\begin{equation}
|G| = \left|\frac{1-4D}{1+4D}\right|
\end{equation}

\noindent Verdt � observere at s� lenge $D > 0 \ \Rightarrow \ |G| \le 1$, slik at Crank-Nicolson-skjemaet er \emph{ubetinget stabilt} 
\end{document}
