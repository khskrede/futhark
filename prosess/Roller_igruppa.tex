\section{Roller i gruppa}
"Ein person blir ein leder når han eller hun blir satt i ein lederposisjon."

I gruppekontrakten blei det bestemt at vi skulle ha en flat struktur i gruppa. 
Valget var basert på at ingen egentlig ville gå inn i ein lederrolle, og at vi 
ville ha ein større frihet til å formulere arbeidsoppgavene selv. 
Vi kom likevel opp i situsjoner som krevde at noen tok ledelse for å sikre 
effektiviteten til gruppa. I praksis fikk gruppa en situasjonsbetinget 
lederstruktur (ref.).

Den situasjonsbetingede lederskapstrukturen førte og til at gruppemedlemmer 
tok rollen som leder i den fagdelen de hadde mest autoritet på. Åsmund ble dermed 
en uformell leder for fysikkdelen, Turid for numerikk delen og Knut for programeringsdelen. 
De tok dermed ansvar for fremgang og oppgavedeling på disse områdene, og strukturerte 
samarbeidet mellom gruppemedlemmene på sitt respektive felt. 

% Åsmund
I vår gruppe så vi at Åsmund i større grad gjikk inn i rolla som oppgave-leder. 
Han tok ansvar for å settje opp eit system for å utveksle informasjon rundt 
prosjektet, github, og tok ofte intiativ ved å dele relevante artikler. 
Han gikk dermed og inn i rolla som samordneren (ref) på gruppa. Vi merket oss 
og at han var mer aktiv i prosjektetdelen av EIT, som er den delen som er mest 
oppgavefokusert. 

% Paul
Bales statuerer at medlemmer som er veldig oppgavefokusert er mindre innvolvert 
i relasjonsrettede aksjoner. Paul er den i gruppen som er mest oppgavefokusert. 
På gruppa har han en tendens til å ta en faglig tilnærming, og liker å fokusere 
mest på oppgava for hand. Han faller alstå oftere inn i rollen som diagnostiker 
(ref) og informasjonssøkande (ref). Han er dermed ein styrke på gruppa ved at han 
holder diskusjonene relevante, i tillegg til at han er eit pålitelig medlem når 
det gjelder å gjennomføre arbeid i tide. Vi ser likevel en tendens til at Paul 
er mindre involvert i relasjonsrettede aksjoner på gruppa.
\\
\\
% Turid
Den som oftest gikk inn i rollen som sosial-emosionell leder var Turid. Hun var et 
viktig ledd i gruppa i form av å uttrykke klare meninger og frustrasjoner over 
moment i gruppa som ikke fungerte optimalt. Hun gikk dermed inn i rolla som kritiker 
og følelsestolker (ref). Når det var ting som fungerte bra i gruppa kom hun og inn 
med støtte og oppmuntring til å fortsette på samme linje. Selv om hun var en av de 
som oftest var innvolvert i diskusjoner/konflikter, var hun også ofte den som løste de, 
enten gjennom forhandlinger eller ved å trekke seg tilbake når hun gikk for langt. 
Turid var og det medlemmet i gruppa som oftest tok initiativ og ledelse i prossessdelen av EIT, 
som fokuserer mest på relasjoner og samarbeid mellom medlemmene i gruppa. Her virker hun 
ofte som målvakt (ref), ved å igangsette runde rundt bordet og være oppmerksom på at alle kom til ordet.
Situasjon: Paul og Turid.
\\
\\
Bale's statuerer og at sosial-emosjonelle aksjoner ofte blir igangsatt av medlemmer som er 
mindre involvert i oppgaveretta aksjoner. Dette ser vi igjen i gruppa ved at Joakim var 
innvolvert i flere sosial-emosjonelle aksjoner. Denne situasjonen har oppstått ved at 
prosjektet har havnet noe utenfor hans fagfelt. Han er dermed blitt mindre involvert i 
sjelve oppgaveløsningen. Derimot har han vært et viktig tillegg i gruppa når det gjelder 
det sosiale samholdet i gruppa. Han har vært den personen som i størst grad har bidratt 
til den lette tonen mellom medlemmene, og har tilrettelagt for at medlemmene i gruppa har 
blitt bedre kjent. Dette har han gjort blant annet ved å alltid stille med en positiv 
innstilling og involvere gruppa i diskusjoner som går på andre ting enn kun det faglige.
\\
\\
Det bor en liten kritiker i samtlige medlemmer på gruppa. Vi ser likevel at Knut har tatt 
på seg denne rollen i gruppa i større grad enn andre. Kritikeren er viktig i gruppesamanheng 
ved «systematisk, åpen, støttande og kritisk gransking av sitt og andres bidrag til gruppearbeidet.»
Dette gjøres på en måte slik at medlemmer på gruppen ikke opplever kritikken som en trussel, 
og dermed beholder fokuset på oppgaveløsing. Knut tar også på seg rollen som meningssøkeren, 
ved at han oppfordrer til at gruppemedlemmene skal få sagt sin mening slik at en beslutning
kan tas.

ref // (1) Handbok for gruppearbeidet
