%Overskriftene er kokt rett fra prosess-rapport-presentasjonen. Endre dem.

\chapter{Oppstart og utvikling}

\section{Situasjon ved oppstart}
\lipsum{1-2}

\section{Formulering av problemstillingen}
\lipsum{3-4}

\section{Opprinnelig plan}
\lipsum{5-6}

\section{Regler, avtaler, kontrakt}
Nødvendigheten av å ha klare regler og avtaler innad i en gruppe diskuteres i
mange av pensumartiklene. I \cite{schwartz} nevnes det som et av ni punktene
avgjørende for en effektiv gruppe ``Etabler regler for hvordan avgjørerelser
tas''. \cite{jj} påpeker hvor viktig det er at gruppa skaper seg en felles
identitet, som hvert gruppemedlem kan samles bak. En samarabeidskontrakt, med
regler for håndtering av konflikter etc., fungerer nettopp på dette viset -
samlende. 

Andre landsbydag var det satt av tid til å skrive samarbeidskontrakt. Det ble, i
samsvar med læringsassistentene, satt opp punkter for hvordan gruppa skulle
håndtere konflikter, ta avgjørelser og ikke minst - mål for gruppearbeid og
tidsfrister. Etter noen uker ble kontrakten reforhandlet. Det viste seg at
enkelte punkter var redundante, mens andre måtte legges til. Særs punktet om
kaffemøtene før oppstart har vist seg effektivt, da det tilbyr en uhøytidelig
stemning, hvor synspunkter kan drøftes under en uformell setting. Det ble i
tillegg lagt merke til at gruppas enhetlige oversikt var manglende, og et nytt
punkt vedrørende fremdriftsplaner ble lagt til. Et eksempel på et punkt som
utgikk var at alle oppgaver skulle føres inn på gruppas wiki-side, dette
overlappet til en viss grad med det nye om fremdriftsplan.

Gruppa hadde også lagt merke til at punktet om tildelte oppgaver skulle føres
inn på wiki-siden aldri ble brukt. At kun tre på gruppa (Joakim, Åsmund og Knut)
hadde kjennskap til github (stedet hvor gruppa lagrer informasjon) kan til dels
ta skylden for det, i tillegg påvirket mangelen av en fremdriftsplan hvert
gruppemedlems oppfatning av egne arbeidsoppgaver. Etter som tiden gikk ble det,
forståelig nok, endel murring fra Paul og Turid, som følte at det var vanskelig
å følge utviklingen da de ikke kunne bruke github. Gruppa satte seg da ned en
søndag og hadde innføring i det, med det resultat at hele gruppa endelig kunne
bruke tjenesten fullt ut.

\section{Beslutningsmønstre}
\lipsum{9-10}

\section{Roller vi tok}
\lipsum{9-10}

\section{Tilbakeblikk}
\lipsum{11-12}
