%Overskriftene er kokt rett fra prosess-rapport-presentasjonen. Endre dem.

\chapter{Oppstart og utvikling}
I første landsbydag fikk vi en kort introduksjon av konseptet bak landsbyen Mia.
Vi fikk informasjon om oppgaver som hadde blitt gjennomført tidligere. Blant
annet handlet en av oppgave om alpinski, og en annen om modellering av
matlaging. Samme dag fikk vi tildelt gruppe; Turid, Paul og Åsmund fra Fysikk og
matematikk, Joakim fra Bygg og miljøteknikk, og Knut Halvor fra Datateknikk. Vi
bestemte gruppas navn, det ble Futhark. Resten av første landsbydag ble brukt
til å diskutere hvilket tema vi hadde lyst på. Det var litt klein stemning i
gruppa, men vi ble saktens kjent med hverandre. 

Andre landsbydag ble brukt til videre diskusjon av tema. Ved siden av hadde vi
en øvelse som gikk ut på å kartlegge kunnskapene til de enkelte gruppemedlemmene
i gruppa. Dette ble gjort ved å at medlemmene førte på sine kunnskapsområder på
en trekant med sider ”teoretisk”, ”faglige”, og ”personlige”. Bilde

Med tanke på hva slags tema vi skulle velge, var dette en viktig øvelse siden
det var viktig å vite hva gruppa var i stand til gjøre. Etter en del diskusjoner
og annet prat hadde gruppa blitt bedre kjent, slik at folk torde å komme til
uttrykk for sine meninger. 
 
I tredje landsbydag ble det iverksatt en øvelse som gikk ut på å presentere våre
erfaringer med gruppearbeid. Øvelsen var nyttig ettersom det gjorde det klart
hva slags roller gruppemedlemmene foretrakk å ha i et gruppearbeid. Samtidig ble
gruppa bedre kjent med hverandres arbeidsvaner. Samme dag fikk gruppen i oppgave
å tilegne seg teorien til "Schwarz" grupperegler. Teorien ble utført ved at
hvert enkelt gruppemedlem skulle presentere nevnte teori for hverandre. Av
Schwarz’ grupperegler var kanskje de viktigste etter vår mening de to første; at
man tester sine antakelser og at man deler all relevant informasjon.  Vi fikk
erfare at når man ignorer disse reglene får det konsekvenser for
gruppedynamikken mye raskere enn vi hadde trodd. Det oppstod to spesifikke
konflikter som følge av dette som vi vil nevne her.

Turid sa at hennes beste erfaring med gruppearbeid var fra håndball, noe Joakim
lo av. Det ble litt dårlig stemning, men diskusjonen ble hysjet ned av
fasilitatorene siden vi skulle gjøre individuelle oppgaver. Så emnet lå og
murret under overflaten til vi fikk snakke fritt igjen, da ble konflikten løst;
Turid sluttet av Joakims latter at det var hånlig ment, men det ble avklart at
Joakim ikke mente noe vondt med latteren.

Her er en annen situasjon. Åsmund gjespet gjentatte ganger mens Turid
presenterte sitt tema fra Schwarz’ teorier, men sluttet med det da nestemann
skulle presentere sitt.  Dette ble fort oppklart, Åsmund var trøtt.

Med disse eksemplene kan man se at konfliktene oppstod på grunn av
feiltolkninger; man hadde ikke kartlagt intensjonene bak det oppfattede
budskapet. Dette ble løst gjennom åpen diskusjon og deling av sine intensjoner
bak handlingene.  Etter de første landsbydagene hadde gruppa blitt godt kjent
med hverandre. Det var klart hva de enkelte gruppemedlemmene var god på, og
hvilke roller hver enkelte sannsynligvis ville ha i gruppesammenheng. Dermed var
alt tilrettelagt for gruppa til å velge oppgavetema.

\section{Situasjon ved oppstart}
Til å begynne med vurderte gruppa "ski" som tema for oppgaven. Ettersom Åsmund
stod mye på ski var dette en oppgave som interesserte ham veldig. De andre i
gruppa som ikke hadde så mye kunnskap om dette temaet var litt mer usikre. Det
ble diskutert de forskjellige problemene som kunne oppstå hvis vi valgte dette
som tema. Særlig Turid var kritisk, og det ble klart at det var flere
strukturelle utfordringer knyttet til denne oppgaven. De fleste i gruppa var
skeptiske fordi oppgaven virket komplisert og uklart definert. Gruppa kom til
slutt fram til at det var mange potensielle problemer som kunne oppstå hvis vi
valgte denne oppgaven.

Ny oppgave som ble foreslått skulle ha noe med mat å gjøre, da dette var et av
de tre hovedtemaene som ble presentert på første landsbydag. Etter en lang
diskusjon av problemstilling kom noen av medlemmene på temaet steking av bacon i
mikrobølgeovn. Dette var en oppgave som skapte stor entusiasme blant alle
gruppemedlemmene. Vi kom frem til at det var få kompliserte utfordringer knyttet
til denne problemstillingen. Oppgaven virket som en spennende ide også fordi det
passet bra med gruppas kunnskaper. Det virket som om oppgaven inkluderte en del
numerikk og fysisk forståelse, noe som passet med de fleste gruppemedlemmene.
Oppgaven inkluderte også visualisering av resultatet, noe som passet med Knut
Halvor fordi han hadde drevet med noe lignende tidligere. Derfor endte det hele
med at vi valgte dette som tema i prosjektet.

Videre ble det anskaffet en del litteratur slik at vi kunne finne ut hva vi
kunne gjøre. Vi fant ut at det fantes en del publikasjoner rundt dette temaet.
Mye tid ble derfor brukt til å finne litteratur til å gi oss en ide om hvordan
vi skal gå løs på oppgaven. For å lette på samarbeidet mellom gruppemedlemmene
ble det satt opp en konto slik at vi kunne dele kode og annet informasjon via
internettet. Da oppgaven var klar skulle gruppen spissen inn på
problemstillingen ytterligere.

I fjerde landsbydag fikk gruppen i oppgave å presentere oppgaven vi hadde valgt,
og litt hvordan vi skulle løse den. Dette medførte til at gruppen ble mer
fokusert på hva vi skulle gjøre. Ettersom det begynte å bli en del ting å gjøre,
satt tok Turid en aksjon og delte ut oppgaver gruppemedlemmene skulle gjøre til
neste dag. Dette gjorde hun fordi hun var frustrert over manglende initiativ.
Med presentasjonen av problemstilling ble det klart hva gruppa skulle gjøre, hva
som var gruppas hovedmål, og hva som kunne gjøres videre hvis vi skulle få
ekstra tid.


\section{Formulering av problemstillingen}

\section{Opprinnelig plan}

\section{Regler, avtaler, kontrakt}
\label{sec:kontrakt}
Nødvendigheten av å ha klare regler og avtaler innad i en gruppe diskuteres i
mange av pensumartiklene. I Schwarz' ``Ground rules for Effective Groups''
\cite{schwarz} nevnes det som et av ni punkter avgjørende for en effektiv
gruppe: ``Etabler regler for hvordan avgjørerelser
tas''. Johnson \& Johnson påpeker i ``Valuing Diversity'' \cite{jj} hvor viktig det er at gruppa skaper seg en felles
identitet, som hvert gruppemedlem kan samles bak. En samarbeidskontrakt, med
regler for håndtering av konflikter, hvordan avgjørelser tas, straff for
regelbrudd etc., fungerer nettopp på dette viset -
samlende. $\\$

Andre landsbydag var det satt av tid til å skrive samarbeidskontrakt. Det ble, i
samsvar med læringsassistentene, satt opp punkter for hvordan gruppa skulle
håndtere konflikter, ta avgjørelser og ikke minst - mål for gruppearbeid og
tidsfrister. Etter noen uker ble kontrakten reforhandlet. Det viste seg at
enkelte punkter var redundante, mens andre måtte legges til. Særs punktet om
kaffemøtene før oppstart har vist seg effektivt, da det tilbyr en uhøytidelig
setting hvor synspunkter kan deles og diskuteres. Det ble i
tillegg lagt merke til at gruppemedlemmenes individuelle oversikt var manglende, og et nytt
punkt vedrørende fremdriftsplaner ble lagt til. Et eksempel på et punkt som
utgikk var at alle oppgaver skulle føres inn på gruppas wiki-side, dette
overlappet til en viss grad med det nye om fremdriftsplan. $\\$

Gruppa hadde også lagt merke til at punktet om tildelte oppgaver skulle føres
inn på wiki-siden aldri ble brukt. At kun tre på gruppa (Joakim, Åsmund og Knut)
hadde kjennskap til github (stedet hvor gruppa lagrer informasjon) kan til dels
ta skylden for det, i tillegg påvirket mangelen av en fremdriftsplan hvert
gruppemedlems oppfatning av egne arbeidsoppgaver. Etter som tiden gikk ble det,
forståelig nok, endel murring fra Paul og Turid, som følte at det var vanskelig
å følge utviklingen. Det ble derfor avtalt at gruppa skulle møtes førstkommende
søndag, slik at samtlige gruppemedlemmer skulle få en innføring i, samt lære seg
bruken av github. 

\section{Beslutningsmønstre}
I gruppekontrakten (se Appendiks XXXX) står det at beslutninger fortrinnsvis
skal tas basert på konsensus, mens det ved splid skal være flertallsavgjørelse.
Gruppa var tidlig innstilt på konsensusavgjørelser, da det sørger for gode og
grundige diskusjoner ved uenighet, og kan i den forstand virke veldig samlende
for en gruppe. Likevel innså vi at enkelte situasjoner kunne bli fullstendig
fastlåst, slik at flertallet måtte få bestemme. Det er likevel viktig at
konsensus har blitt \emph{forsøkt} oppnådd før flertallsavgjørelse
implementeres. $\\$

Den første avgjørelsen gruppa tok var valg av oppgave. Åsmund var i starten
veldig giret på skiforsøket. Som gikk ut på å måle spenstforfallet i en
slalomski utover sesongen. Resten av gruppa var noe mer moderate i
begeistringen. Det ble derfor arrangert ``høring'' hvor samtlige gruppemedlemmer
kom med forslag. Disse ble så samlet og diskutert, og gruppa forsøkte å finne en
oppgave hvor alle kunne bidra. Valget falt derfor på steking av bacon i
microbølgeovn. Da dette hadde et massivt innslag av numerikk, programmering og
fysikk. Disipliner gruppa dekte meget godt mellom seg. $\\$

I samarbeidskontrakten er gruppestrukturen betegnet som ``riddere av det runde
bord''. Dette er et prinsipp som anvendes i alle våre beslutninger. Når et
problem oppstår går ordet rundt bordet, og hvert gruppemedlem sier sin mening.
Det diskuteres deretter i plenum, og om mulig oppnås en konsensusavgjørelse. Vi
har kun ved ett tilfelle anvendt flertall-paragrafen i kontrakten og det var i
forbindelse med språkvalg på prosessrapporten. Knut var i utgangspunktet uvillig
til å skrive på norsk, og det gikk ikke å ``vinne'' han over. Etter noen
minutter med diskusjon ble altså Knut Halvor overstyrt.

\subsection{Roller i gruppa}
"Ein person blir ein leder når han eller hun blir satt i ein lederposisjon."

%" Lederskap begynner med ein formel rolle struktur som definerer gruppe hierakiet av autoritet.(Authority; legitimate power assigned to a particular position.)

I gruppekontrakten blei det bestemt at me skulle ha ein flatstruktur i gruppa. Valget var basert på at ingen egentlig ville gå inn i ein lederrolle, og at me ville ha ein større frihet til å formulere arbeidsoppgåvene sjølv. Me kom likevel opp i situsjoner som krevde at ein person tok ledelse for å sikre effektiviteten til gruppa. \emph{her må vi ha et eksempel} I praksis fekk gruppa ein situasjonsbetinget lederstruktur (ref.).\\

Den situasjonsbetinget lederskap strukturen førte og til at gruppemedlemmer tok rollen som leder i den fagdelen dei hadde mest autoritet på. Åsmund blei dermed ein uformell leder for fysikkdelen, Turid for numerikk delen og Knut for programeringsdelen i c++. Dei tok dermed ansvar for framgang og oppgåvedeling på desse områdene, og strukturerte samarbeidet mellom gruppemedlemmene på sitt respektive felt. \\

I vår gruppe såg me at Åsmunde i større grad gjekk inn i rolla som oppgåve-leder. Han tok ansvar for å settje opp eit system for å utveksle informasjon rundt prosjektet, github, og tok ofte intiativ ved å leggje ut relevante artikler på sida. Han gjekk dermed og inn i rolla som  samordneren (ref) på gruppa. Me merka oss og at han var meir aktiv i prosjektet i EIT, som er den delen som er mest oppgåvefokusert. \\

Bales staturer at medlemmer som er veldig oppgåvefokusert er mindre innvolvert i relasjonsretta aksjoner. Paul er den i gruppa som er mest oppgåvefokusert.På gruppa har han ein tendens til å ta ein faglig tilnærming, og liker å fokusere mest på oppgåva for hand. Han fell alstå oftere inn i rolla som diagnostiker (ref) og informasjonssøkjande (ref). Han er dermed ein styrke på gruppa ved at han held diskusjonene relevante, i tillegg til at han er eit pålitelig medlem når det gjeld å gjennomføre arbeid i tide. Me ser likevel ein tendens til at Paul er mindre involvert i relasjonsretta aksjoner på gruppa. \\

Den som oftas gjekk inn i rolla som sosial-emosionell leder var Turid. Ho var eit viktig ledd i gruppa i form av å uttrykke klare meininger og frustrasjoner over moment i gruppa som ikkje fungerte optimalt. Ho gjekk dermed inn i rolla som kritiker og kjensletolker (ref). Når det var ting som fungerte bra i gruppa kom ho og inn med støtte og oppmuntring til å fortsette på samme linje. Sjølv om ho var ein av dei som oftest var innvolvert i diskusjoner/konflikter, var ho og ofte den som løyste dei, enten gjennom forhandlinger eller ved å trekkje seg tilbake når ho gjekk for langt. Turid var og det medlemmet i gruppa som oftes tok initiativ og ledelse i prossessdelen av EIT, som fokuserer mest på relasjoner og samarbeid mellom medlemmer på gruppa. Her verka ho ofte som målvakt (ref), ved å igangsette runde rundt bordet og være oppmerksom på at alle kom til ordet.
Situasjon: Paul og Turid.\\

Bale's statuerer og at sosial-emotionell aksjoner ofte blir igangsatt av medlemmer som er mindre involvert i oppgåveretta aksjoner. Dette ser me igjen i gruppa ved at Joakim var innvolvert fleire sosial-emosjonelle aksjoner. Denne situasjonen har oppstått ved at prosjektet har havnet noko utenfor hans fagfelt. Han er dermed blitt mindre involvert i sjølve oppgåveløysninga. Derimot har han vore eit viktig tillegg i gruppa når det gjeld det sosiale samhaldet i gruppa. Han har vore den personen som i størst grad har bidratt til den lette tonen mellom medlemmene, og har tilrettelagt for at medlemmene i gruppa har blitt betre kjent. Dette har han gjort blant anna ved å alltid stille med ein positiv innstilling og involvere gruppa i diskusjoner som går på andre ting enn kun det faglige.\\

Det bur ein liten kritikker i samtlige medlemmer på gruppa. Me ser likevel at Knut har tatt på seg denne rolla i gruppa i større grad enn andre. Kritikkaren er viktig i gruppesamanheng ved «systematisk, åpen, støttande og kritisk gransking av sitt og andres bidrag til gruppearbeidet.»
Dette gjøres på ein måte slik at medlemmer på gruppa ikkje opplever kritikken som ein trussel, og dermed beholder fokuset på oppgåveløysning. (Innlegg frå andre).\\

ref // (1) Handbok for gruppearbeidet\\

\section{Tilbakeblikk}

