%Overskriftene er kokt rett fra prosess-rapport-presentasjonen. Endre dem.

\chapter{Oppstart og utvikling}

\section{Situasjon ved oppstart}
\lipsum{1-2}

\section{Formulering av problemstillingen}
\lipsum{3-4}

\section{Opprinnelig plan}
\lipsum{5-6}

\section{Regler, avtaler, kontrakt}
Nødvendigheten av å ha klare regler og avtaler innad i en gruppe diskuteres i
mange av pensumartiklene. I Schwarz' ``Ground rules for Effective Groups''
\cite{schwarz} nevnes det som et av ni punkter avgjørende for en effektiv
gruppe: ``Etabler regler for hvordan avgjørerelser
tas''. Johnson \& Johnson påpeker i ``Valuing Diversity'' \cite{jj} hvor viktig det er at gruppa skaper seg en felles
identitet, som hvert gruppemedlem kan samles bak. En samarbeidskontrakt, med
regler for håndtering av konflikter, hvordan avgjørelser tas, straff for
regelbrudd etc., fungerer nettopp på dette viset -
samlende. $\\$

Andre landsbydag var det satt av tid til å skrive samarbeidskontrakt. Det ble, i
samsvar med læringsassistentene, satt opp punkter for hvordan gruppa skulle
håndtere konflikter, ta avgjørelser og ikke minst - mål for gruppearbeid og
tidsfrister. Etter noen uker ble kontrakten reforhandlet. Det viste seg at
enkelte punkter var redundante, mens andre måtte legges til. Særs punktet om
kaffemøtene før oppstart har vist seg effektivt, da det tilbyr en uhøytidelig
setting hvor synspunkter kan deles og diskuteres. Det ble i
tillegg lagt merke til at gruppemedlemmenes individuelle oversikt var manglende, og et nytt
punkt vedrørende fremdriftsplaner ble lagt til. Et eksempel på et punkt som
utgikk var at alle oppgaver skulle føres inn på gruppas wiki-side, dette
overlappet til en viss grad med det nye om fremdriftsplan. $\\$

Gruppa hadde også lagt merke til at punktet om tildelte oppgaver skulle føres
inn på wiki-siden aldri ble brukt. At kun tre på gruppa (Joakim, Åsmund og Knut)
hadde kjennskap til github (stedet hvor gruppa lagrer informasjon) kan til dels
ta skylden for det, i tillegg påvirket mangelen av en fremdriftsplan hvert
gruppemedlems oppfatning av egne arbeidsoppgaver. Etter som tiden gikk ble det,
forståelig nok, endel murring fra Paul og Turid, som følte at det var vanskelig
å følge utviklingen. Det ble derfor avtalt at gruppa skulle møtes førstkommende
søndag, slik at samtlige gruppemedlemmer skulle få en innføring i, samt lære seg
bruken av github. 

\section{Beslutningsmønstre}
I gruppekontrakten (se Appendiks XXXX) står det at beslutninger fortrinnsvis
skal tas basert på konsensus, mens det ved splid skal være flertallsavgjørelse.
Gruppa var tidlig innstilt på konsensusavgjørelser, da det sørger for gode og
grundige diskusjoner ved uenighet, og kan i den forstand virke veldig samlende
for en gruppe. Likevel innså vi at enkelte situasjoner kunne bli fullstendig
fastlåst, slik at flertallet måtte få bestemme. Det er likevel viktig at
konsensus har blitt \emph{forsøkt} oppnådd før flertallsavgjørelse
implementeres. $\\$

Den første avgjørelsen gruppa tok var valg av oppgave. Åsmund var i starten
veldig giret på skiforsøket. Som gikk ut på å måle spenstforfallet i en
slalomski utover sesongen. Resten av gruppa var noe mer moderate i
begeistringen. Det ble derfor arrangert ``høring'' hvor samtlige gruppemedlemmer
kom med forslag. Disse ble så samlet og diskutert, og gruppa forsøkte å finne en
oppgave hvor alle kunne bidra. Valget falt derfor på steking av bacon i
microbølgeovn. Da dette hadde et massivt innslag av numerikk, programmering og
fysikk. Disipliner gruppa dekte meget godt mellom seg. $\\$

I samarbeidskontrakten er gruppestrukturen betegnet som ``riddere av det runde
bord''. Dette er et prinsipp som anvendes i alle våre beslutninger. Når et
problem oppstår går ordet rundt bordet, og hvert gruppemedlem sier sin mening.
Det diskuteres deretter i plenum, og om mulig oppnås en konsensusavgjørelse. Vi
har kun ved ett tilfelle anvendt flertall-paragrafen i kontrakten og det var i
forbindelse med språkvalg på prosessrapporten. Knut var i utgangspunktet uvillig
til å skrive på norsk, og det gikk ikke å ``vinne'' han over. Etter noen
minutter med diskusjon ble altså Knut Halvor overstyrt.
\section{Roller vi tok}
\lipsum{9-10}

\section{Tilbakeblikk}
\lipsum{11-12}
