%Overskriftene er kokt rett fra prosess-rapport-presentasjonen. Endre dem.

\chapter{Oppstart og utvikling}

\section{Situasjon ved oppstart}
\lipsum{1-2}

\section{Formulering av problemstillingen}
\lipsum{3-4}

\section{Opprinnelig plan}
\lipsum{5-6}

\section{Regler, avtaler, kontrakt}
Nødvendigheten av å ha klare regler og avtaler innad i en gruppe diskuteres i
mange av pensumartiklene. I Schwarz' ``Ground rules for Effective Groups''
\cite{schwarz} nevnes det som et av ni punkter avgjørende for en effektiv
gruppe: ``Etabler regler for hvordan avgjørerelser
tas''. Johnson \& Johnson påpeker i ``Valuing Diversity'' \cite{jj} hvor viktig det er at gruppa skaper seg en felles
identitet, som hvert gruppemedlem kan samles bak. En samarbeidskontrakt, med
regler for håndtering av konflikter, hvordan avgjørelser tas, straff for
regelbrudd etc., fungerer nettopp på dette viset -
samlende. $\\$

Andre landsbydag var det satt av tid til å skrive samarbeidskontrakt. Det ble, i
samsvar med læringsassistentene, satt opp punkter for hvordan gruppa skulle
håndtere konflikter, ta avgjørelser og ikke minst - mål for gruppearbeid og
tidsfrister. Etter noen uker ble kontrakten reforhandlet. Det viste seg at
enkelte punkter var redundante, mens andre måtte legges til. Særs punktet om
kaffemøtene før oppstart har vist seg effektivt, da det tilbyr en uhøytidelig
setting hvor synspunkter kan deles og diskuteres. Det ble i
tillegg lagt merke til at gruppemedlemmenes individuelle oversikt var manglende, og et nytt
punkt vedrørende fremdriftsplaner ble lagt til. Et eksempel på et punkt som
utgikk var at alle oppgaver skulle føres inn på gruppas wiki-side, dette
overlappet til en viss grad med det nye om fremdriftsplan. $\\$

Gruppa hadde også lagt merke til at punktet om tildelte oppgaver skulle føres
inn på wiki-siden aldri ble brukt. At kun tre på gruppa (Joakim, Åsmund og Knut)
hadde kjennskap til github (stedet hvor gruppa lagrer informasjon) kan til dels
ta skylden for det, i tillegg påvirket mangelen av en fremdriftsplan hvert
gruppemedlems oppfatning av egne arbeidsoppgaver. Etter som tiden gikk ble det,
forståelig nok, endel murring fra Paul og Turid, som følte at det var vanskelig
å følge utviklingen. Det ble derfor avtalt at gruppa skulle møtes førstkommende
søndag, slik at samtlige gruppemedlemmer skulle få en innføring i, samt lære seg
bruken av github. 

\section{Beslutningsmønstre}
I gruppekontrakten (se Appendiks XXXX) står det at beslutninger fortrinnsvis
skal tas basert på konsensus, mens det ved splid skal være flertallsavgjørelse.
Gruppa var tidlig innstilt på konsensusavgjørelser, da det sørger for gode og
grundige diskusjoner ved uenighet, og kan i den forstand virke veldig samlende
for en gruppe. Likevel innså vi at enkelte situasjoner kunne bli fullstendig
fastlåst, slik at flertallet måtte få bestemme. Det er likevel viktig at
konsensus har blitt \emph{forsøkt} oppnådd før flertallsavgjørelse
implementeres. $\\$

Den første avgjørelsen gruppa tok var valg av oppgave. Åsmund var i starten
veldig giret på skiforsøket. Som gikk ut på å måle spenstforfallet i en
slalomski utover sesongen. Resten av gruppa var noe mer moderate i
begeistringen. Det ble derfor arrangert ``høring'' hvor samtlige gruppemedlemmer
kom med forslag. Disse ble så samlet og diskutert, og gruppa forsøkte å finne en
oppgave hvor alle kunne bidra. Valget falt derfor på steking av bacon i
microbølgeovn. Da dette hadde et massivt innslag av numerikk, programmering og
fysikk. Disipliner gruppa dekte meget godt mellom seg. $\\$

I samarbeidskontrakten er gruppestrukturen betegnet som ``riddere av det runde
bord''. Dette er et prinsipp som anvendes i alle våre beslutninger. Når et
problem oppstår går ordet rundt bordet, og hvert gruppemedlem sier sin mening.
Det diskuteres deretter i plenum, og om mulig oppnås en konsensusavgjørelse. Vi
har kun ved ett tilfelle anvendt flertall-paragrafen i kontrakten og det var i
forbindelse med språkvalg på prosessrapporten. Knut var i utgangspunktet uvillig
til å skrive på norsk, og det gikk ikke å ``vinne'' han over. Etter noen
minutter med diskusjon ble altså Knut Halvor overstyrt.
\section{Roller vi tok}
``Ein person blir ein leder når han eller hun blir satt i ein lederposisjon.''

%" Lederskap begynner med ein formel rolle struktur som definerer gruppe hierakiet av autoritet.(Authority; legitimate power assigned to a particular position.)

I gruppekontrakten blei det bestemt at me skulle ha ein flatstruktur i gruppa. Valget var basert på at ingen egentlig ville gå inn i ein lederrolle, og at me ville ha ein større frihet til å formulere arbeidsoppgåvene sjølv. Me kom likevel opp i situsjoner som krevde at ein person tok ledelse for å sikre effektiviteten til gruppa. I praksis fekk gruppa ein situasjonsbetinget lederstruktur.

\subsection{Litt teori}
Situsjonsbetinget lederskap er definert som delt lederskap blant gruppemedlemmer der medlemmene varierer oppførselen etter kva funksjon gruppa treng til ein kvar tid. Ein funksjon er ein aksjon som blir sett igang for å sikre effektiviteten i gruppa. Me kan snakke om to typer aksjoner; oppgåveretta aksjoner og relasjonsretta/samholdsretta (relationship) aksjoner. 

I Robert-Bales interaksjon prosess analyse trekk han fram at i ein gruppe er det ofte nokon som intrer i rolla som oppgåve-leder og ein som trer inn i rolla som sosial-emosiell leder. Kjenneteiknet på ein oppgåve-leder er at det er ofte han som setter igang aksjoner for å sikre at målet til gruppa blir nådd (oppgåveretta aksjoner), mens den sosiall-emosielle lederen setter igang aksjoner for å oppretthalde og forbedre interrelasjonene i gruppa (relasjonsretta aksjoner).

\subsection{Gruppa}
I vår gruppe såg me at Åsmunde i større grad gjekk inn i rolla som oppgåve-leder. Han tok ansvar for å settje opp eit system for å utveksle informasjon rundt prosjektet, github, og tok ofte intiativ ved å leggje ut relevante artikler på sida. Han var og meir aktiv i prosjektet i EIT, som er den delen som er mest oppgåvefokusert. 
Bales staturer og at medlemmer som er veldig oppgåvefokusert er mindre innvolvert i relasjonsretta aksjoner. Paul er den i gruppa som er mest oppgåvefokusert.På gruppa har han ein tendens til å ta ein faglig tilnærming, og liker å fokusere mest på oppgåva for hand. Han er dermed ein styrke på gruppa ved at han held diskusjonene relevante, i tillegg til at han er eit pålitelig medlem når det gjeld å gjennomføre arbeid i tide. Me ser likevel ein tendens til at Paul er mindre involvert i relasjonsretta aksjoner på gruppa. 

Den som oftas gjekk inn i rolla som sosial-emosionell leder var Turid. Ho var eit viktig ledd i gruppa i form av å uttrykke klare meininger og frustrasjoner over moment i gruppa som ikkje fungerte optimalt. Når det var ting som fungerte bra i gruppa kom ho inn med støtte og oppmuntring til å fortsette på samme linje. Sjølv om ho var ein av dei som oftest var innvolvert i diskusjoner/konflikter, var ho og ofte den som løyste dei, enten gjennom forhandlinger eller ved å trekkje seg tilbake når ho gjekk for langt. Turid var og det medlemmet i gruppa som oftes tok initiativ og ledelse i prossessdelen av EIT, som fokuserer mest på relasjoner og samarbeid mellom medlemmer på gruppa. 
Situasjon: Paul og Turid.

Bale's statuerer og at sosial-emotionell aksjoner ofte blir igangsatt av medlemmer som er mindre involvert i oppgåveretta aksjoner. Dette ser me igjen i gruppa ved at Joakim har innført fleire sosial-emosjonelle aksjoner. Denne situasjonen har oppstått ved at prosjektet har havnet noko utenfor hans fagfelt. Han er dermed blitt mindre involvert i sjølve oppgåveløysninga. Derimot har han vore eit viktig tillegg i gruppa når det gjeld det sosiale samhaldet i gruppa. Han har vore den personen som i størst grad har bidratt til den lette tonen mellom medlemmene, og har tilrettelagt for at medlemmene i gruppa har blitt betre kjent. Dette har han gjort blant anna ved å alltid stille med ein positiv innstilling og involvere gruppa i diskusjoner som går på andre ting enn kun det faglige.


Når det gjeld å løse opp spenninga på gruppa, spela alle medlemmene på gruppa ei viktig rolle ved å leggje ann ein uformell tone i gruppa.
Humor!


Den situasjonsbetinget lederskap strukturen førte og til at gruppemedlemmer tok rollen som leder i den fagdelen dei hadde mest autoritet på. Åsmund blei dermed ein uformell leder for fysikkdelen, Turid for numerikk delen og Knut for programeringsdelen i c++. Dei tok dermed ansvar for framgang og oppgåvedeling på desse områdene, og strukturerte samarbeidet mellom gruppemedlemmene på desse områdene. 

\section{Tilbakeblikk}
\lipsum{11-12}
