\chapter{Presentasjon av gruppens medlemmer}
\person{Åsmund}
Åsmund går i 4.klasse på studieretningen Teknisk Fysikk, med fordypning 
i matematisk fysikk. Åsmund har jobbet med numerikk og fysikk på sin forrige 
sommerjobb, og i tillegg gjør hans gode generelle kompetanse i fysikk at han 
er godt skikket til å finne de riktige differensiallikningene som beskriver 
det fysiske systemet som skal studeres.

\subsection*{Innstilling til gruppearbeid}
Åsmund har hatt både positive og negative erfaringer med gruppearbeid tidligere.
Han har merket seg at de gruppene han har negative erfaringer fra ofte har hatt
medlemmer som ikke har vært motiverte for arbeidsoppgaven. Dette har delvis
motivert hans valg av MiA som landsby, da han tror at de som velger MiA er
innstilte på å gjøre en god jobb.

\subsection*{Gruppa om Åsmund}

Åsmund er \ldots

\person{Joakim}
Jeg studerer sivilingeniør Bygg- og miljøteknikk, med
spesialiseringen innenfor konstruksjonsteknikk og beregningsmekanikk. Fra før
har jeg erfaring med teamarbeid gjennom førsteklassefaget Fysisk
Miljøplanlegging, som på lik linje med EiT hadde en prosjektrapport og
prosessrapport som evalueringsform.

\subsection*{Innstilling til gruppearbeid}

Jeg liker å jobbe etter klare og veldefinerte problemer. Likevel
er jeg en stor tilhenger av en situasjonsbetinget lederstruktur, hvor gruppen
tilpasser seg underveis i arbeidet. På den måten, mener jeg, at hvert
gruppemedlem blir tildelt den rollen som faller han/hun naturlig, der og da. 

Fra før har jeg hatt ett rent prosjektfag, der var gruppen preget av
homogenitet, da samtlige var fra samme linje. Det er derfor ekstra spennende og
jobbe i en såpass heterogen gruppe, med hhv. en fysiker (Åsmund), programmerer
(Knut), numeriker (Turid), statistiker (Paul) og meg (konstruktør). Jeg har
utelukkende gode erfaringer med gruppearbeid, da jeg føler at jeg yter mer når
andre er avhengig av jobben jeg gjør, og vice versa.

Min posisjon i gruppa er mindre selvsagt enn de resterende medlemmene. Da jeg har generell
kunnskap om de fleste disipliner, har de inngående kunnskap i hvert sitt felt.
Slik at jeg har tatt litt del i det meste, noe jeg liker godt. Jeg prøver derfor
å legge til rette for et godt gruppemiljø, da jeg anser det som vel så viktig.

\subsection*{Gruppa om Joakim}

%\subsection*{Grunnlag for landsbyvalg og forventninger til EiT}

%\subsection*{Påvirker forventninger og kompetanse gruppearbeidet?}

%\subsection*{Personlige forutsetninger}

\person{Turid}
\lipsum[3-4]

\person{Paul}
Paul går i 4.klasse på studieretningen fysikk og matematikk, med spesialising i statistikk. Han har erfaring fra gruppearbeid fra prosjektoppgaver i statistikk, numerikk og matematisk modellering.

\subsection*{Innstilling til gruppearbeid}
Paul liker å ha oversikt over arbeidsoppgaver og at gruppa har en framdriftsplan. Han er nøye med det han gjør, og står på når han har fått en konkret arbeidsoppgave. Han er flink til å lytte til andres meninger, og kommer på ting som han kan bidra med i gruppa.

\subsection*{Gruppa om Paul}

\person{Knut Halvor}
\lipsum[7-8]

