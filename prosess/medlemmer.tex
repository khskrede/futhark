\chapter{Presentasjon av gruppens medlemmer}
\person{Åsmund}
Åsmund går i 4.klasse på studieretningen Teknisk Fysikk, med fordypning 
i matematisk fysikk. Åsmund har jobbet med numerikk og fysikk på sin forrige 
sommerjobb, og i tillegg gjør hans gode generelle kompetanse i fysikk at han 
er godt skikket til å finne de riktige differensiallikningene som beskriver
det fysiske systemet som skal studeres.

\subsection*{Innstilling til gruppearbeid}
Åsmund har hatt både positive og negative erfaringer med gruppearbeid tidligere.
Han har merket seg at de gruppene han har negative erfaringer fra ofte har hatt
medlemmer som ikke har vært motiverte for arbeidsoppgaven. Dette har delvis
motivert hans valg av MiA som landsby, da han tror at de som velger MiA er
innstilte på å gjøre en god jobb.

\subsection*{Gruppa om Åsmund}

Åsmund er gruppas fysiker. Han er flink til å ta initiativ og tar på seg mye.
Det er meget lett å spørre Åsmund, da han er imøtekommende, ressurssterk og
hjelpende av natur. I gruppearbeidet er han meget aktiv, men kan bli oppslukt i enkeltoppgaver og
melder seg da litt ut av gruppa. Han har en rik form for humor som favner
mye, noe som bidrar til god og avslappet stemning på gruppa. Åsmund er vårt
ordensmenneske, som har oversikt over de ulike trådene og syr de sammen til det
endelige resultate, både hva angår prosess og prosjekt.

\person{Joakim}
Jeg studerer sivilingeniør Bygg- og miljøteknikk, med
spesialiseringen innenfor konstruksjonsteknikk og beregningsmekanikk. Fra før
har jeg erfaring med teamarbeid gjennom førsteklassefaget Fysisk
Miljøplanlegging, som på lik linje med EiT hadde en prosjektrapport og
prosessrapport som evalueringsform.

\subsection*{Innstilling til gruppearbeid}

Jeg liker å jobbe etter klare og veldefinerte problemer. Likevel
er jeg en stor tilhenger av en situasjonsbetinget lederstruktur, hvor gruppen
tilpasser seg underveis i arbeidet. På den måten, mener jeg, at hvert
gruppemedlem blir tildelt den rollen som faller han/hun naturlig, der og da. 

Fra før har jeg hatt ett rent prosjektfag, der var gruppen preget av
homogenitet, da samtlige var fra samme linje. Det er derfor ekstra spennende og
jobbe i en såpass heterogen gruppe, med hhv. en fysiker (Åsmund), programmerer
(Knut), numeriker (Turid), statistiker (Paul) og meg (konstruktør). Jeg har
utelukkende gode erfaringer med gruppearbeid, da jeg føler at jeg yter mer når
andre er avhengig av jobben jeg gjør, og vice versa.

Min posisjon i gruppa er mindre selvsagt enn de resterende medlemmene. Da jeg har generell
kunnskap om de fleste disipliner, har de inngående kunnskap i hvert sitt felt.
Slik at jeg har tatt litt del i det meste, noe jeg liker godt. Jeg prøver derfor
å legge til rette for et godt gruppemiljø, da jeg anser det som vel så viktig.

\subsection*{Gruppa om Joakim}
Joakim er gruppas humørspreder, og sørger for god stemning også når dagen blir
lang og gruppa er sliten. Joakims egentlige fagbakgrunn er litt på siden av
gruppas problemstilling, men han har vært veldig flink til å finne oppgaver som
han får gjort. Han har også mye kompetanse på områder som han ikke har drevet så
mye med, f.eks. numerikk og statistikk. Det at Joakim er en humørspreder gjør at 
han av og til sporer av, men dette er i all hovedsak positivt for gruppa. Selv om 
det går ut over den faglige produktiviteten der og da, har det bidratt til at 
gruppa har blitt mye bedre kjent, og sannsynligvis også til å øke produktiviteten 
på lang sikt.

\person{Turid}
Studerer ved industriell matematikk ved linja numerisk matematikk ved NTNU, med hovedvekt på numerisk matematikk. Har tidligere gått på folkehøgskole ved linja søm og design og handball. Var med på UKA-06 som kostymedesigner for revyen.

\subsection*{Innstilling til Gruppearbeid}
Jeg har tidligere jobbet en del i grupper gjennom studiet på NTNU. Disse gruppene har midlertidig vært prega av en homogenitet i forhold til fag bakgrunn. Det har gjort at vi angriper nye problemstillinger med en lik tankegang, og vi mister evnen til å komme opp med mer kreative ideer til løsningsmetoder. De differansene som jeg har opplevd i samband av gruppearbeid, har ofte grunna i ulike personligheter og kommunikasjonstil. Da vi ikke har hatt noe opplæring i å takle ulike differanser og konflikter som disse ofte fører til, har disse differansene ofte hatt en negativ innvirkning på gruppearbeidet i form av uløste konflikter og stagnering. Jeg har dermed sett fram til muligheten med å lære mer om sjølve gruppeprosessen og gruppedynamikk for å kunne takle slike situasjoner annerledes i framtida. Jeg trives vanligvis godt i grupper, og synes gruppearbeid kan være både fremende og utfordrende.  Jeg såg derfor fram til å jobbe i ei mer heterogen gruppe, med tanke på fag bakgrunn.


%\subsection*{Turid om sin rolle i gruppa}
%
%Jeg valgte landbyen MIA utifra min interesse for numerisk matematikk. Jeg falt dermed ganske naturlig inn i ei rolle med ansvar for den numeriske berekninger i prosjektet. Når det gjaldt arbeide rundt gruppeprosessen merka jeg fort at jeg fekk en noe kontrollerende og styrende rolle. Jeg følte selv at det var ei nødvendig  rolla for å holde effektiviteten og fokuset oppe i denne delen av gruppearbeidet. Jeg føler dermed at denne rollen i noe grad var ei rolle som blei tilegna meg av gruppa. Som eneste jente på gruppe merket jeg og en større forventing til meg når det gjaldt refleksjoner over gruppearbeidet. Jeg fekk dermed ofte rollen som observatør av gruppa, både i prosjektarbeid og ulike prosess øvinger. 

\subsection*{Gruppa om Turid}
Turid er ansvarsbevisst og har tatt mye initiativ for prosessdelen. Hun er en
samlende faktor i gruppa, som kan hente oss inn når vi sporer av. Turid har vært
involvert i mange av konfliktene på gruppa, muliggens fordi hun er jente,
men også fordi hun gjerne vil få igjennom viljen sin. Turid kaller en spade for
en spade, noe resten av gruppa setter pris på. Turid har vært aktiv i de aller
fleste diskusjoner, og er flink til å delegere oppgaver og å passe på at alle
har noe å gjøre.

\person{Paul}
Paul går i 4.klasse på studieretningen fysikk og matematikk, med spesialising i statistikk. Han har erfaring fra gruppearbeid fra prosjektoppgaver i statistikk, numerikk og matematisk modellering.

\subsection*{Innstilling til gruppearbeid}
Paul liker å ha oversikt over arbeidsoppgaver og at gruppa har en framdriftsplan. Han er nøye med det han gjør, og står på når han har fått en konkret arbeidsoppgave. Han er flink til å lytte til andres meninger, og kommer på ting som han kan bidra med i gruppa.

\subsection*{Gruppa om Paul}
Paul er en stillferdig person i gruppesammenheng, men på tomannshånd er han mer
aktiv og vil presentere sin mening. Når en avgjørelse må tas i gruppa må han
involveres mer aktivt enn noen av de andre medlemmene. Paul er flink til å ta 
initiativ på det faglige, og veldig fokusert på oppgaven. Han koordinerer bra 
med de andre medlemmene i forhold til sitt fagfelt, og er et positivt bidrag til
gruppa. Hvis man gir Paul en oppgave, er man sikker på at den blir godt
gjennomført og på tiden.

\person{Knut Halvor}
Knut Halvor studerer 4. året ved NTNU på linjen datateknikk. Studieretningen han 
har valgt er komplekse datasystemer. Tidligere har han hatt både gode og dårlige
erfaringer fra gruppearbeid. Det som går igjen fra gruppene han har hatt gode
erfaringer med er at medlemmene er innstilte på og gjøre en god jobb, og at de
har høye forventniger til hverandre.

\subsection*{Innstilling til gruppearbeid}
Jeg foretrekker og ha klare retningslinjer for hvilke oppgaver de forskjellige 
medlemmene har i gruppen. På denne måten kan oppgaver og løsningsmetoder diskuteres,
uten at de blir "overdiskutert". Selv om han foretrekker å alene ha ansvaret for sine oppgaver,
foretrekker han og jobbe sammen med andre, da diskusjoner rundt problemstillinger og
løsningsmetoder hjelper han å få organisert tankene sine.

\subsection*{Gruppa om Knut Halvor}
Knut Halvor er gruppas datamann, noe som definerer hans forhold til gruppa i
stor grad. Han er flink til å bidra med sin kompetanse i diskusjoner, og står på
sine meninger når man skal ta en avgjørelse som har å gjøre med hans fagfelt. På
mange måter en Knut Halvor den mest faglig kritiske i gruppa. Det at han hovedsaklig
jobber med programmering og er veldig selvstendig, er avgjørende for kvaliteten på 
gruppas sluttresultat, men har også gjort at det tok lengre tid før gruppa ble godt 
kjent med ham.

