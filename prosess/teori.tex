\chapter{Teori}
\section{Schwarz - gruppeteori}
\subsection{Effektivitet}


Flere faktorer påvirker en gruppes effektivitet; har gruppen et klart mål? Er
medlemmene enig om arbeidsmetode, og er de motiverte for den? For å forenkle
jobben som fasilitator, og/eller gi gruppemedlemmene mulighet til å påvirke
effektiviteten selv, har Schwarz utviklet en ``modell'' for hva som gjør grupper
effektive. Det er verdt å merke seg at Schwarz siterer George Box i
introduksjonen til kapittelet: ``All models are wrong; some are useful''.

Schwarz setter opp tre kriterier for effektivt gruppearbeid:

\begin{itemize}
\item[\textsc{Ytelse}] Tjenesten eller produktet gruppen leverer møter (eller overgår)
	forventningene og/eller kravene til de som skal bruke/motta/evaluere
	tjenesten.
\item[\textsc{Prosess}] Prosessene bak utførelsen av arbeidet vedlikeholder, eller
forbedrer, gruppemedlemmenes mulighet til å samarbeide på videre
arbeidsoppgaver.
\item[\textsc{Personlig}] Gruppeerfaringen skal bidra til personlig vekst og god
selvfølelse blant medlemmene.
\end{itemize}

Disse kriteriene er avhengige av hverandre, og for at en gruppe skal være
effektiv må den møte alle tre kritierer. Bryter gruppen på ett punkt vil det
påvirke de andre, for eksempel: Hvis en gruppe håndterer konflikter på en slik
måte at tilliten mellom medlemmene avtar, vil det på sikt føre til at medlemmene
holde informasjon tilbake fra resten av gruppen. Dette vil igjen føre til at
nøkkelinformasjon ikke er tilgjengelig for samtlige gruppemedlemmer, og gruppen
vil bli ineffektiv.

Videre beskrives måter en gruppe kan sikre seg effektivitet. En effektiv gruppe
håndterer konflikter i det åpne, og baserer seg på antagelsen om at hver enkelt
medlem er sterk nok til å takle negativ tilbakemelding. I tillegg er det viktig
å ikke bare løse konflikten, men også finne ut hva som forårsaket den. $\\$

Det er også viktig at man kommuniserer på en slik måte at mottaker og avsender
har samme oppfatning av hva som blir sagt. Altså må man passe på at det man sier
blir oppfattet korrekt, dette kan oppnås ved å ikke bare avsløre hva du har
funnet ut, men også \emph{hvordan} du kom til den konklusjonen. Ved å bruke
denne kommunikasjonsmetoden oppmuntres andre til å finne ``hull'' i logikk og
løsningsmetode, i tillegg til at mottaker blir tvunget til å komme med

motargumenter. Effektive grupper tester også ut sine antagelser, det vil si at
dersom et gruppemedlem sier noe svarer mottaker med å fortelle hva han/hun tror
den andre mente. Man minimerer da muligheten for misforståelser og videre
frustrasjon. $\\$

Effektive grupper må også ha klare definerte rammer for oppgaven som skal løses, 
det er også viktig at hvert gruppemedlem kjenner (og klarer å uttrykke) denne
oppgaven. Gruppen må også sørge for at hvert medlem har en klar oppfatning av
sin egen rolle i oppnåelsen av gruppeoppgaven. I samme omgang bør det sørges for
at beslutningsprosesser og hierarki i gruppen er klart definert og forstått. $\\$

En enkel, men god, måte og oppnå alle disse punktene på, er å bruke endel tid i
starten på å få til en bra samarbeidskontrakt. Der hvor tema som lederskap,
problemstilling, beslutningsmønstre, oppgavefordeling osv. kan innlemmes. Se
\cref{sec:kontrakt}. Schwarz har kokt alt dette ned til 9 grunnregler som skal sørge for
gruppeeffektivitet. Se \cref{sec:grunnregler}.
\clearpage

\subsection{Grunnregler}
\label{sec:grunnregler}
For å sikre effektivt gruppearbeid har Schwarz \cite{schwarz} etablert et sett
med grunnleggende regler (\cref{tab:grunnregler}), dette sørger for at gruppa har et enkelt og
oversiktlig oppslagsverk og vise til, for eksempel ved diskusjoner. I tillegg er
det en god pekepinn på punkter som bør være i samarbeidskontrakten.
\begin{center}
\begin{table}[ht!]
\begin{tabular}{r l}
1. & Test antagelser og slutninger \\
2. & Del \emph{all} relevant informasjon \\
3. & Bruk spesifikke eksempler og bli enig om betydning av fremmedord \\
4. & Forklar resonnement og intensjon. \\
5. & Fokuser på interesser, ikke posisjon. \\
6. & Kombiner støtte og granskning \\
7. & Utform fremgangsplaner og tilbakemeldingsmåter i plenum. \\
8. & Etabler metode for å ta opp ``ikke-tema''. \\
9. & Bruk et beslutningsmønster som sørger for nødvendig engasjement. \\
\end{tabular}
\caption{Schwarz' regler for effektive grupper}
\label{tab:grunnregler}
\end{table}
\end{center}

\section{Johnson \& Johnson}
\subsection{Gruppedynamikk}
Gruppedynamikk handler om den vitenskapelige studien av oppførsel i grupper.
Atferd i grupper, og forståelsen av den, er essensielt, da mennesker i all
hovedsak er gruppedyr, det være seg i familien, blant venner eller på jobb. Det
første som må læres er; ``hva er en gruppe?''. Spørsmålet er vanskeligere enn
man skulle tro, og Johnson \& Johnson (J\&J) \cite{jj} viser til at sosiologene
fortsatt ikke er enige om én definisjon. Likevel kan man si at en gruppe er en
samling av minimun to personer, som relaterer med hverandre ansikt til ansikt.
Samtlige individer i gruppa er også klar over at de er avhengige av de andre for
å få jobben gjort, i tillegg til at det er klart definert \emph{hvem} som er med
i gruppa. $\\$

J\&J påstår videre at en gruppes produktivitet er avhengig av fem
basiselementer:
\begin{itemize}
\item[$1.$] Positiv avhengighet blant medlemmer.
\item[$2.$] Individuell ansvarsfølelse.
\item[$3.$] Fremmende interaksjon.
\item[$4.$] Fungerende sosiale antenner.
\item[$5.$] Gode gruppeprosesser.
\end{itemize}
Hvis man i tillegg skal sørge for at gruppen er effektiv må gruppemedlemmene
\textbf{(1)} Sørge for at gruppens felles mål anvender hvert medlems
individuelle kompetanse, \textbf{(2)} forsikre at kommunikasjon mellom hverandre
er nøyaktig og forstås korrekt, \textbf{(3)} etablere lederskap og/eller
passende innflytelse, \textbf{(4)} implementere beslutningsmønstre som sørger
for at samtlige medlemmer kommer med innslag, i tillegg til at hver enkelts
resonnement og konklusjon er åpne for evaluering og analysering, \textbf{(5)} 


% Lurer på om dette ikke er en forlengelse av de syv grunnreglene. Mulig det er
% smart å skrive de to samlet.
