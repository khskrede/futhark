\chapter{Teori}
\section{Schwarz - gruppeteori}
\subsection{Effektivitet}


Flere faktorer påvirker en gruppes effektivitet; har gruppen et klart mål? Er
medlemmene enig om arbeidsmetode, og er de motiverte for den? For å forenkle
jobben som fasilitator, og/eller gi gruppemedlemmene mulighet til å påvirke
effektiviteten selv, har Schwarz utviklet en ``modell'' for hva som gjør grupper
effektive. Det er verdt å merke seg at Schwarz siterer George Box i
introduksjonen til kapittelet: ``All models are wrong; some are useful''.

Schwarz setter opp tre kriterier for effektivt gruppearbeid:

\begin{itemize}
\item[\textsc{Ytelse}] Tjenesten eller produktet gruppen leverer møter (eller overgår)
	forventningene og/eller kravene til de som skal bruke/motta/evaluere
	tjenesten.
\item[\textsc{Prosess}] Prosessene bak utførelsen av arbeidet vedlikeholder, eller
forbedrer, gruppemedlemmenes mulighet til å samarbeide på videre
arbeidsoppgaver.
\item[\textsc{Personlig}] Gruppeerfaringen skal bidra til personlig vekst og god
selvfølelse blant medlemmene.
\end{itemize}

Disse kriteriene er avhengige av hverandre, og for at en gruppe skal være
effektiv må den møte alle tre kritierer. Bryter gruppen på ett punkt vil det
påvirke de andre, for eksempel: Hvis en gruppe håndterer konflikter på en slik
måte at tilliten mellom medlemmene avtar, vil det på sikt føre til at medlemmene
holde informasjon tilbake fra resten av gruppen. Dette vil igjen føre til at
nøkkelinformasjon ikke er tilgjengelig for samtlige gruppemedlemmer, og gruppen
vil bli ineffektiv.

Videre beskrives måter en gruppe kan sikre seg effektivitet. En effektiv gruppe
håndterer konflikter i det åpne, og baserer seg på antagelsen om at hver enkelt
medlem er sterk nok til å takle negativ tilbakemelding. I tillegg er det viktig
å ikke bare løse konflikten, men også finne ut hva som forårsaket den. $\\$

Det er også viktig at man kommuniserer på en slik måte at mottaker og avsender
har samme oppfatning av hva som blir sagt. Altså må man passe på at det man sier
blir oppfattet korrekt, dette kan oppnås ved å ikke bare avsløre hva du har
funnet ut, men også \emph{hvordan} du kom til den konklusjonen. Ved å bruke
denne kommunikasjonsmetoden oppmuntres andre til å finne ``hull'' i logikk og
løsningsmetode, i tillegg til at mottaker blir tvunget til å komme med

motargumenter. Effektive grupper tester også ut sine antagelser, det vil si at
dersom et gruppemedlem sier noe svarer mottaker med å fortelle hva han/hun tror
den andre mente. Man minimerer da muligheten for misforståelser og videre
frustrasjon. $\\$

Effektive grupper må også ha klare definerte rammer for oppgaven som skal løses, 
det er også viktig at hvert gruppemedlem kjenner (og klarer å uttrykke) denne
oppgaven. Gruppen må også sørge for at hvert medlem har en klar oppfatning av
sin egen rolle i oppnåelsen av gruppeoppgaven. I samme omgang bør det sørges for
at beslutningsprosesser og hierarki i gruppen er klart definert og forstått. $\\$

En enkel, men god, måte og oppnå alle disse punktene på, er å bruke endel tid i
starten på å få til en bra samarbeidskontrakt. Der hvor tema som lederskap,
problemstilling, beslutningsmønstre, oppgavefordeling osv. kan innlemmes. Se
\cref{sec:kontrakt}. Schwarz har kokt alt dette ned til 9 grunnregler som skal sørge for
gruppeeffektivitet. Se \cref{sec:grunnregler}.
\clearpage

\subsection{Grunnregler}
\label{sec:grunnregler}
For å sikre effektivt gruppearbeid har Schwarz \cite{schwarz} etablert et sett
med grunnleggende regler (\cref{tab:grunnregler}), dette sørger for at gruppa har et enkelt og
oversiktlig oppslagsverk og vise til, for eksempel ved diskusjoner. I tillegg er
det en god pekepinn på punkter som bør være i samarbeidskontrakten.
\begin{center}
\begin{table}[ht!]
\begin{tabular}{r l}
1. & Test antagelser og slutninger \\
2. & Del \emph{all} relevant informasjon \\
3. & Bruk spesifikke eksempler og bli enig om betydning av fremmedord \\
4. & Forklar resonnement og intensjon. \\
5. & Fokuser på interesser, ikke posisjon. \\
6. & Kombiner støtte og granskning \\
7. & Utform fremgangsplaner og tilbakemeldingsmåter i plenum. \\
8. & Etabler metode for å ta opp ``ikke-tema''. \\
9. & Bruk et beslutningsmønster som sørger for nødvendig engasjement. \\
\end{tabular}
\caption{Schwarz' regler for effektive grupper}
\label{tab:grunnregler}
\end{table}
\end{center}

\section{Johnson \& Johnson}
\subsection{Gruppedynamikk}
Gruppedynamikk handler om den vitenskapelige studien av oppførsel i grupper.
Adferd i grupper, og forståelsen av den, er essensielt, da mennesker i all
hovedsak er gruppedyr, det være seg i familien, blant venner eller på jobb. Det
første som må læres er; ``hva er en gruppe?''. Spørsmålet er vanskeligere enn
man skulle tro, og Johnson \& Johnson (J\&J) \cite{jj} viser til at sosiologene
fortsatt ikke er enige om én definisjon. Likevel kan man si at en gruppe er en
samling av minimun to personer, som relaterer med hverandre ansikt til ansikt.
Samtlige individer i gruppa er også klar over at de er avhengige av de andre for
å få jobben gjort, i tillegg til at det er klart definert \emph{hvem} som er med
i gruppa. $\\$

J\&J påstår videre at en gruppes produktivitet er avhengig av fem
basiselementer:
\begin{itemize}
\item[$1.$] Positiv avhengighet blant medlemmer.
\item[$2.$] Individuell ansvarsfølelse.
\item[$3.$] Fremmende interaksjon.
\item[$4.$] Fungerende sosiale antenner.
\item[$5.$] Gode gruppeprosesser.
\end{itemize}
Hvis man i tillegg skal sørge for at gruppen er effektiv må gruppemedlemmene
\textbf{(1)} Sørge for at gruppens felles mål anvender hvert medlems
individuelle kompetanse, \textbf{(2)} forsikre at kommunikasjon mellom hverandre
er nøyaktig og forstås korrekt, \textbf{(3)} etablere lederskap og/eller
passende innflytelse, \textbf{(4)} implementere beslutningsmønstre som sørger
for at samtlige medlemmer kommer med innslag, i tillegg til at hver enkelts
resonnement og konklusjon er åpne for evaluering og analysering, \textbf{(5)}
etabler metoder for å håndtere konflikter konstruktivt.$\\$

\subsection{Å verdsette ulikheter}
Det er uunngåelig at en gruppe vil bestå av personer med ulike egenskaper. Det
være seg faglige, personlige, kulturelle, religiøse, etniske og så videre.
Det er viktig å sørge for at ulikhetene blant gruppemedlemmene kulminerer i noe
positivt, Johnson \& Johnson presenterer ulike måter man kan forsikre dette på.
Man må forsikre at det mellom gruppemedlemmene er en stor grad av positiv
avhengighet av hverandres arbeid. Gruppen må også forsøke å etablere en felles
gruppeidentitet som alle medlemmene kan samle seg bak, denne må være basert på
flertallets verdier. I oppstartsfasen er det viktig at det brukes tid på å lære
om forskjellene mellom medlemmene, et slikt enkelt grep kan kartlegge ulike
måter og ordlegge seg på (eksempelvis i ulike kulturer), som kan forhindre
miskommunikasjon og misforståelser på et senere stadium. $\\$

Det er også viktig at det etableres gode rutiner for konflikthåndtering, noe som
helt sikkert vil oppstå i heterogene grupper. Konfliktene bør håndteres slik at
gruppa får klarhet i hva som forårsaket uenigheten, og hva som ble gjort for å
løse den. Bruk også tid på at gruppemedlemmene skal bli kjent med hverandre på
et personlig nivå, på denne måten forsikrer man at diskusjonene mellom personene
i gruppa blir friere og bedre.

\section{Wheelan}
ubsection{Effective teammedlemmer}

Wheeland beskriver gruppemedlemmer's oppførsel og holdninger i effektive grupper. En viktig del ved å være en effektiv gruppemedlem består i å vurdere sin egen oppførsel og holdning og måten man kommuniserer med gruppa på. Teorien er presentert i form av retningslinjer. 

\begin{itemize}
\item[1.] Ikke skyld på andre for gruppas problemer
\item[2.] Vær engasjert når gruppa setter mål, roller og avklaring av oppgaver
\item[3.] Vær for en åpen kommunikasjonsstruktur der alle medlemmer deltar og blir hørt 
\item[4.] Ha en riktig fordeling mellom diskusjon av oppgaven og støttende kommunikasjon
\item[5.] Bruk en effektiv måte å løse oppgaver på, og en effektiv måte å ta avgjørelser
\item[6.] Lage normer som støtter produktivitet, innovasjon og frihet til å uttale seg
\item[7.] Bidra med normer som forfremmer produktivitet
\item[8.] Forfremme gruppearbeid
\end{itemize}

Punkt 1 kaller Wheeland for ''the fundamental attribution error'', fordi vi ofte tilskriver andres holdninger til personlige trekk uten å ta andre faktorer i betraktning. For eksempel kan vi skylde på sjefen for dårlige resultater uten å ta i betraktning budsjettrestriksjoner, og dårlig gruppesammarbeid. En gruppe blir ikke effektiv før alle tar ansvar for gruppa's samarbeid og produktivitet. \\

Punkt 2 dreier seg om å være klar over hva som foregår i gruppearbeidet. Hvis man ikke forstår hva som foregår, må man våge å spørre. Å stille spørsmål til hele gruppa vil føre til en rikere diskusjon og vil klargjøre ting for alle gruppemedlemmer. \\

Punkt 3 tar opp problemet med at medlemmer i gruppa ikke tør å si det de mener fordi de ofte klassifiserer andre i gruppa, og ut i fra det angir dem høyere eller lavere status. Disse medlemmene kan bli ignonert i gruppa, noe som fører til at gruppa blir mindre produktiv. En måte å sikre at alle blir hørt på er så enkelt som å ta en runde rundt bordet for å høre hva hver og en har å si. \\

Punkt 4 handler om oppgaveorientert diskusjon i gruppen. Forskning tyder på at suksessfulle grupper bruker om lag 70-80 \% av tiden sin til å diskutere oppgave og mål. Men ofte hender det at gruppa går seg bort i en lang diskusjon om noe annet. \\

Punkt 5 diskuterer effektive måter for oppgaveløsning og det å ta beslutninger. Det nevnes fire steg som kan følges: Å innse problemet, diagnotisere problemet, ta beslutningen, og akseptere og gjennomføre beslutningen. I praksis består de to første stegene i å planlegge strategier for å løse problemet. Når man så skal ta en beslutning er det ikke noe klart svar på hvilken beslutningsmåte som er best, men gruppa skal velge en beslutningsmåte der alle medlemmer kan akseptere beslutningen.  \\

Punkt 6 handler om gruppas avtale. Hvis gruppas medlemmer bare er middelsmådige enige, blir resultatet også middelsmådig. Hvis medlemmene derimot er enige om å gjøre en best mulig jobb og løse problemer på best mulig måte, er det større sannsynlighet for at resultatet blir bra. Frihet til å uttale seg er nevnt i punkt 3. \\

Punkt 7 dreier seg om normer, dvs. regler om medlemmenes oppførsel og hva som skal gjøres. Normer er nødvendige for å samordne gruppas arbeid mot et felles mål. Spesielt er normer viktige når medlemmene er uenige i hvordan ting skal gjøres. \\

Punkt 8 diskuterer karaktertrekk ved god gruppesamarbeid. Disse er bl.a. god kommunikasjon, vennlig atmosfære, stor innsatsvilje, og stor grad av arbeidsfordeling.

\section{Kommunikasjonsteori}
%
%look at patterns of group communication and the variables that influence communication effectiveness
%
For at ei gruppe skal kunne fungere effektivt, er det viktig at medlemmer klarer å kommuniserer enkelt og effektivt. Særlig i en tværrfaglig gruppe er det viktig at medlemmer er effektive i å formidle videre den informasjonen dei innehar. (Då løysninga av problemet i stor grad avhenger av evnen til gruppa har til å flette sammen informasjon som fleire medlemmer innehar, er det viktig at mottakeren av informasjon sitter igjen med det samme bilde som sender prøver å formidle.) Det er derfor viktig å he ein grunnleggende forståelse for gruppekommikasjon.
For å forstå gruppekommunikasjonen i eit team, er det to ting som er viktige å sjå nærmere på; kommunikasjonsmønsteret/ kommunikasjonsnettverk og kommunikasjonseffektiviteten. Ved kommunikasjonsmønsteret meiner me her kven som kommunniserer med kven. Me kan her snakke om ein einvegskommunikajsone, der lederen adresserer gruppa, eller ein toveiskommunikasjon, som t.d. myldring blant gruppemedlemmer. Kommunikasjonseffektiven er her definert som evnen til å formidle ein beskjed på ein slik måte at mottakeren tolker beskjeden slik som sender sjølv har tenkt.
Variable som kan verke inn på gruppe kommunikasjonen er normer på gruppa, humor osv.
\\
\\
Det kommunikasjonsnettverka som blei nytta på gruppa vår opent nettverk og sirkel nettverk. I eit opent nettverk kommuniserer alle med alle. Denne formen for kommunikasjon kjem tydelig fram i diskusjoner på gruppa der det er fri flyt av ideer og meininger. Når ein avgjørelse blir tatt på gruppa starter me ofte med ein open diskusjon, og utvekslinger av meininger. Før ein avgjørelse blir tatt har me vanligvis hatt ein runde rundt bordet der alle får uttrykkt sin meining om saken. Dette er ein form for sirkel nettverk, der eit medlem uttrykker sin meining før han sender ordet videre til sidemannen. Dette er ein kommunikasjonsform som har utviklet seg naturlig på gruppa. Særlig i prosessdelen har det falt naturlig å ta ein runde rundt bordet for å høyre dei ulike medlemmenes meininger od følelser. Det er ofte Turid som har tatt initiativ til runde rundt bordet, og har dermed falt inn i ein rolle som ordstyrer. 
\\
\section{Roller}
«I grupparbete spelar rollerna roll» (Stefan Jern, (1))\\

«Ei rolle kan defineres som ein spesiell type særegenhet, som definerer din plass på gruppa». Ei gruppe består av ulike individ som kvar har sine særegenskaper. Desse personlige trekka er ein faktor som er med på å definere kva rolle eit medlem vil spele på gruppa.  Når ei gruppe går sammen, byggjes det forventninger til rolla dei andre vil ha på gruppa utfrå inntrykket dei får av individet. Dine egne forventninger og gruppas forventninger til di rolle er i stor grad med på å forme den rolla du vil ha i gruppa vidare. I eit gruppesamarbeid med flat struktur snakker me her om informelle roller. Informelle roller er roller som vekser fram gjennom samspelet i gruppa. Informelle roller i eit gruppesamarbeid kan i hovudsak deles i to roller; instrumentelle roller og sosio-emosjonelle roller. 
\\
\\
I Robert-Bales likevektsteori statuerer han viktigheten ved ein balanse mellom arbeidsoppgåver og sosial-emosjonelle aktiviteter i effektive grupper. I effektive grupper ser ein derfor ofte at eit medlem trer inn i rolla som instrumentell leder, mens ein annan trer inn i rolla som sosial-emosiell leder. Kjenneteiknet på ein oppgåve-leder er at det er ofte han som setter igang aksjoner for å sikre at målet til gruppa blir nådd (oppgåveretta aksjoner), mens den sosiall-emosielle lederen setter igang aksjoner for å oppretthalde og forbedre interrelasjonene i gruppa (relasjonsretta aksjoner).
\\
\\
I ein flatstrukturert gruppe er lederrollen ofte avhengig av situasjoner gruppa møter. Situsjonsbetinget lederskap er definert som delt lederskap blant gruppemedlemmer der medlemmene varierer oppførselen etter kva funksjon gruppa treng til ein kvar tid. Ein funksjon er ein aksjon som blir sett igang for å sikre effektiviteten i gruppa, og kan vera enten oppgåveretta aksjoner og relasjonsretta/samholdsretta aksjoner. 

Benne og Sheat beskreiv i 1948, tretten instumentelle roller. Her tar eg for meg nokon av dei som var mest fremmtredene i vår gruppe; initiativtakeren, informasjonssøkaren, informasjonsgiveren, meiningsøkjaren, meiningsgivaren, diagnostikeren, samordnaren, opplysnaren, energigiveren og kritikkeren. Desse er alle viktige roller på gruppa. 

Meiningsøkjaren spelar ein viktig del på gruppa ved at han innbyr til beslutningstaking, og legg grunnlaget for individ til å fremleggje sin vurdering av framlagte fakta. Diagnostikeren er den analyserende parten på gruppa, som tar opp problemer/utfordringer ved oppgåve, og eventuelt leder gruppa i ei ny retning. Samordnaren derimot bidrar ved å samkjøre gruppas arbeid og flette dei samman til ein heilhet.

Benne og Sheat satt og opp åtte sosio-emosjonelle roller i ei gruppe; oppmuntraren, harmoniseraren, spenningoppløseren, kompromisten, målvakten, kjenslertolken, normsetteren og følgeren. Ein målvakt verkar som ein slags ordstyrar, som øker kommunikasjonen ved å bremse pratmakerene og passe på at alle kjem til ordet. Desse rollene kan flukturerer mellom medlemmene på gruppa, men ofte ser ein at eit individ påtar seg nokon roller med høgare frekvens.
