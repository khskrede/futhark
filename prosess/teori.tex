\chapter{Teori}
\section{Schwarz - gruppeteori}
\subsection{Effektivitet}


Flere faktorer påvirker en gruppes effektivitet; har gruppen et klart mål? Er
medlemmene enig om arbeidsmetode, og er de motiverte for den? For å forenkle
jobben som fasilitator, og/eller gi gruppemedlemmene mulighet til å påvirke
effektiviteten selv, har Schwarz utviklet en ``modell'' for hva som gjør grupper
effektive. Det er verdt å merke seg at Schwarz siterer George Box i
introduksjonen til kapittelet: ``All models are wrong; some are useful''.

Schwarz setter opp tre kriterier for effektivt gruppearbeid:

\begin{itemize}
\item[\textsc{Ytelse}] Tjenesten eller produktet gruppen leverer møter (eller overgår)
	forventningene og/eller kravene til de som skal bruke/motta/evaluere
	tjenesten.
\item[\textsc{Prosess}] Prosessene bak utførelsen av arbeidet vedlikeholder, eller
forbedrer, gruppemedlemmenes mulighet til å samarbeide på videre
arbeidsoppgaver.
\item[\textsc{Personlig}] Gruppeerfaringen skal bidra til personlig vekst og god
selvfølelse blant medlemmene.
\end{itemize}

Disse kriteriene er avhengige av hverandre, og for at en gruppe skal være
effektiv må den møte alle tre kritierer. Bryter gruppen på ett punkt vil det
påvirke de andre, for eksempel: Hvis en gruppe håndterer konflikter på en slik
måte at tilliten mellom medlemmene avtar, vil det på sikt føre til at medlemmene
holde informasjon tilbake fra resten av gruppen. Dette vil igjen føre til at
nøkkelinformasjon ikke er tilgjengelig for samtlige gruppemedlemmer, og gruppen
vil bli ineffektiv.

Videre beskrives måter en gruppe kan sikre seg effektivitet. En effektiv gruppe
håndterer konflikter i det åpne, og baserer seg på antagelsen om at hver enkelt
medlem er sterk nok til å takle negativ tilbakemelding. I tillegg er det viktig
å ikke bare løse konflikten, men også finne ut hva som forårsaket den. $\\$

Det er også viktig at man kommuniserer på en slik måte at mottaker og avsender
har samme oppfatning av hva som blir sagt. Altså må man passe på at det man sier
blir oppfattet korrekt, dette kan oppnås ved å ikke bare avsløre hva du har
funnet ut, men også \emph{hvordan} du kom til den konklusjonen. Ved å bruke
denne kommunikasjonsmetoden oppmuntres andre til å finne ``hull'' i logikk og
løsningsmetode, i tillegg til at mottaker blir tvunget til å komme med
motargumenter.

% Lurer på om dette ikke er en forlengelse av de syv grunnreglene. Mulig det er
% smart å skrive de to samlet.
