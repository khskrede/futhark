\documentclass[a4paper,norsk]{article}
\usepackage[utf8]{inputenc}
\usepackage[T1]{fontenc}
\usepackage{babel, textcomp}
\usepackage{graphicx}
\usepackage{fancyvrb}
\usepackage{amsmath}
\usepackage{multirow}
\usepackage[margin=1.5in]{geometry}
\setlength{\parindent}{0in}

\tolerance = 10000
\hbadness = \tolerance
\pretolerance = 2000
\author{Joakim Johnsen - joakij@stud.ntnu.no}

\begin{document}

\section{Presentasjon av gruppemedlemmer}

\subsection{Joakim}

Jeg studerer sivilingeniør Bygg- og miljøteknikk, med
spesialiseringen innenfor konstruksjonsteknikk og beregningsmekanikk. Fra før
har jeg erfaring med teamarbeid gjennom førsteklassefaget Fysisk
Miljøplanlegging, som på lik linje med EiT hadde en prosjektrapport og
prosessrapport som evalueringsform.

\subsubsection*{Innstilling til gruppearbeid}

Jeg liker å jobbe etter klare og veldefinerte problemer. Likevel
er jeg en stor tilhenger av en situasjonsbetinget lederstruktur, hvor gruppen
tilpasser seg underveis i arbeidet. På den måten, mener jeg, at hvert
gruppemedlem blir tildelt den rollen som faller han/hun naturlig, der og da. 

Fra før har jeg hatt ett rent prosjektfag, der var gruppen preget av
homogenitet, da samtlige var fra samme linje. Det er derfor ekstra spennende og
jobbe i en såpass heterogen gruppe, med hhv. en fysiker (Åsmund), programmerer
(Knut), numeriker (Turid), statistiker (Paul) og meg (konstruktør). Jeg har
utelukkende gode erfaringer med gruppearbeid, da jeg føler at jeg yter mer når
andre er avhengig av jobben jeg gjør, og vice versa.

Min posisjon i gruppa er mindre selvsagt enn de resterende medlemmene. Da jeg har generell
kunnskap om de fleste disipliner, har de inngående kunnskap i hvert sitt felt.
Slik at jeg har tatt litt del i det meste, noe jeg liker godt. Jeg prøver derfor
å legge til rette for et godt gruppemiljø, da jeg anser det som vel så viktig.

\subsubsection*{Gruppen om Joakim}

%\subsection*{Grunnlag for landsbyvalg og forventninger til EiT}

%\subsection*{Påvirker forventninger og kompetanse gruppearbeidet?}

%\subsection*{Personlige forutsetninger}

\end{document} 
