\chapter{Konklusjon}
Allerede på de første landsbydagene la vi merke til at vi hadde hatt flaks med
gruppesammensetningen. Alle gruppas medlemmer var innstilte på å gjøre en god
jobb, og vi merket fort at stemningen i gruppa var god. På starten var vi
noe negativt innstilt til jobbing med prosess. Vi så ikke på forhånd nytten
av å dedikere spesifikk tid til å arbeide med teambygging, og mente at teambygging
ville skje naturlig dersom vi bare fikk jobbe med prosjektet. \\

Etterhvert som vi fikk jobbet noen ganger med prosjekt og med prosess så vi at
prosessarbeidet har en klar nytteverdi. Øvelsene har i seg selv vært ganske
gode, selv om det har vært litt murring, hovedsaklig fordi det har avbrutt oss
mens vi var godt igang med prosjektarbeidet. Det at vi har arbeidet med prosess
samt å tilegne oss teori har ført til at grupperegler har ligget i bakhodet når
situasjoner har oppstått, slik at vi har tatt teorien i bruk i praksis, f.eks.
når vi .\\

Vi har også merket at vår tilegnelse av teorien har gjort det enklere å håndtere
situasjoner som har med gruppestruktur og gruppedynamikk å gjøre. Når man kan henvise
til en bestemt teori som underbygger argumenter for å omstrukturere er det
enklere å bli enige om et felles ståsted.\\

Den største endringen i gruppedynamikken så vi under framføringen i midten av
semesteret. Medlemmene fikk inntrykk av at de andre på gruppa var oppmerksomme
når man snakket, og at man fikk god støtte hvis man ble utrygg. Dette gjorde at
flere av medlemmene følte seg trygge nok til å improvisere med eksempler rundt
gruppas framgang. \\


