\section{Teori roller}
«I grupparbete spelar rollerna roll» (Stefan Jern, (1))

«Ei rolle kan defineres som ein spesiell type særegenhet, som definerer din plass på gruppa». Ei gruppe består av ulike individ som kvar har sine særegenskaper. Desse personlige trekka er ein faktor som er med på å definere kva rolle eit medlem vil spele på gruppa.  Når ei gruppe går sammen, byggjes det forventninger til rolla dei andre vil ha på gruppa utfrå inntrykket dei får av individet. Dine egne forventninger og gruppas forventninger til di rolle er i stor grad med på å forme den rolla du vil ha i gruppa vidare. I eit gruppesamarbeid med flat struktur snakker me her om informelle roller. Informelle roller er roller som vekser fram gjennom samspelet i gruppa. Informelle roller i eit gruppesamarbeid kan i hovudsak deles i to roller; instrumentelle roller og sosio-emosjonelle roller. 
\\
\\
I Robert-Bales likevektsteori statuerer han viktigheten ved ein balanse mellom arbeidsoppgåver og sosial-emosjonelle aktiviteter i effektive grupper. I effektive grupper ser ein derfor ofte at eit medlem trer inn i rolla som instrumentell leder, mens ein annan trer inn i rolla som sosial-emosiell leder. Kjenneteiknet på ein oppgåve-leder er at det er ofte han som setter igang aksjoner for å sikre at målet til gruppa blir nådd (oppgåveretta aksjoner), mens den sosiall-emosielle lederen setter igang aksjoner for å oppretthalde og forbedre interrelasjonene i gruppa (relasjonsretta aksjoner).
\\
\\
I ein flatstrukturert gruppe er lederrollen ofte avhengig av situasjoner gruppa møter. Situsjonsbetinget lederskap er definert som delt lederskap blant gruppemedlemmer der medlemmene varierer oppførselen etter kva funksjon gruppa treng til ein kvar tid. Ein funksjon er ein aksjon som blir sett igang for å sikre effektiviteten i gruppa, og kan vera enten oppgåveretta aksjoner og relasjonsretta/samholdsretta aksjoner. 

Benne og Sheat beskreiv i 1948, tretten instumentelle roller. Her tar eg for meg nokon av dei som var mest fremmtredene i vår gruppe; initiativtakeren, informasjonssøkaren, informasjonsgiveren, meiningsøkjaren, meiningsgivaren, diagnostikeren, samordnaren, opplysnaren, energigiveren og kritikkeren. Desse er alle viktige roller på gruppa. 

Meiningsøkjaren spelar ein viktig del på gruppa ved at han innbyr til beslutningstaking, og legg grunnlaget for individ til å fremleggje sin vurdering av framlagte fakta. Diagnostikeren er den analyserende parten på gruppa, som tar opp problemer/utfordringer ved oppgåve, og eventuelt leder gruppa i ei ny retning. Samordnaren derimot bidrar ved å samkjøre gruppas arbeid og flette dei samman til ein heilhet.

Benne og Sheat satt og opp åtte sosio-emosjonelle roller i ei gruppe; oppmuntraren, harmoniseraren, spenningoppløseren, kompromisten, målvakten, kjenslertolken, normsetteren og følgeren. Ein målvakt verkar som ein slags ordstyrar, som øker kommunikasjonen ved å bremse pratmakerene og passe på at alle kjem til ordet. Desse rollene kan flukturerer mellom medlemmene på gruppa, men ofte ser ein at eit individ påtar seg nokon roller med høgare frekvens.