\chapter{Introduction}
\section{Problem formulation}
The problem considered in this project is the preparation of bacon in a microwave oven. Bacon is
defined as cured meat from side and back cuts of a pig. A microwave oven in this context is a
household appliance typically delivering 750W of microwave effect at a frequency of 2.1 GHz
to a Faraday cage of volume 200 liters. 

The motivation for the problem formulation was clear. As soon as modeling of food preparation was
brought on the table, bacon seemed a prime choice. A microwave oven may seem an odd choice for
bacon, but previous experiments have shown that optimal bacon can consistently be attained with this
preparation method. Obvious advantages over traditional bacon preparation include less cleaning up
to do afterwards and a shorter time-to-plate. The caveat is that microwave preparation is less
suitable for feedback during the cooking process, as the bacon is obscured from view. To the
traditional bacon chef, used to a touch-and-go approach, this is an obstacle to implementation, as
overcooking bacon in a microwave oven yields inedible results. The
purpose of this work, then, is to establish reliable numerical simulations that can serve as a basis
for estimating cooking time for an arbitrary slice of bacon.

As the project considers the optimal preparation of bacon in a microwave oven, two natural questions
were formulated:

\begin{itemize}
  \item How is preparation of bacon in a microwave oven modelled numerically?
  \item How does water and fat content affect the preparation, and what preparation time is optimal?
\end{itemize}

The first question is obvious, as any attempt at predicting the preparation time necessitates a
numerical model of bacon in a microwave oven; a simple glance at the relevant equations 


Det første spørsmålet var naturlig å stille, for om baconet skal være optimalt stekt, må vi kunne
beskrive hvordan baconet utvikler seg i en mikrobølgeovn. Dette kan da hjelpe oss med å finne ut
fort baconet blir stekt. Det andre spørsmålet er avgjørende for å få til et optimalt bacon. Siden
det er mange typer bacon er det nødvendig å finne ut hvordan fett-, vann- og saltinnhold påvirker
stekingen, fordi forskjellige typer bacon naturligvis utvikler seg annerledes under steking. Også
hvilken effekt vi satte mikrobølgeovnen på vil nødvendigvis påvirke stekingen. En effekt som
mikrobølgeovnen gir er ikke nødvendigvis lik den effekten som mottas av baconet. Egenskapene til
mikrobølgeovnen er derfor en parameter i modellen. 

Når vi forsøker å få svar på disse spørsmålene er det naturlig å sette opp hva som er input og hva
som er output i modellen. Som input har vi allerede nevnt fett-, vann- og saltinnholdet i baconet.
Dette er input som vi kan få direkte fra baconpakningen. I tillegg er tykkelsen og geometrien av
baconet viktige størrelser. Om vi skal steke mange baconskiver på en gang, vil det sannsynligvis ha
betydning for stekingen, så antallet baconskiver vil også være en input. Effekten vi setter på
mikrobølgeovnen vil være en input i modellen fordi det vil ha stor betydning på resultatet. Til
slutt kan vi nevne ”ønsket sprøhet” som en mulig inputparameter fordi forskjellige folk har
forskjellige meninger om hva som er det optimale bacon. Noen ønsker å ha baconet mer saftig, mens
andre vil ha det sprøest mulig. Når inputparametrene er satt inn i modellen, trenger den bare å
finne ut hvor lang tid vi trenger å steke baconet. Eneste output i modellen er derfor steketiden.

For å modellere steking av bacon i mikrobølgeovn, har vi tatt i betraktning hvordan baconet blir
varmet opp, og hvordan fett- og vanninnholdet i baconet forandrer seg over tid. Åpenbare ligninger
som kan brukes for å beskrive disse prosessene er varmeligningen og ligningen for massetransport.
Med varmeligningen i tre dimensjonen vil vi kunne beskrive hvor fort baconet varmes opp.
Temperaturen i baconet er nært knyttet til når det brunes opp. Med massetransportligningen vil vi
finne ut hvor mye masse som trekkes ut av baconet. Den kan gjøre det mulig å finne kritisk massetap.
Sammen vil ligningene ha betydning for hvor fort baconet blir stekt. Gitt varmeligningen og
massetransportligningen er det også nødvendig å sette opp nøyaktige grensebetingelser for baconet i
en mikrobølgeovn for at modellen skal være realistisk. Hvis modellen skulle beskrive prosessen i
baconet godt nok, vil den kunne gi svar på hvor lang tid vi trenger å steke baconet gitt
inputparametrene i modellen. For å verifisere at modellen er god, må det gjøres eksperimenter med
samme inputparametre. De virkelige resultatene kan da brukes til å sammenligne med beregningene gitt
av modellen. Dette er hovedmålet i oppgaven. 

Videre er det mulig å få modellen mer realistisk ved å sette elektromagnetiske grensebetingelser i
modellen. Hvis alle disse målene nås innen god tid, vil gruppa også forsøke å lage en app for den
reduserte modellen som kan gi svar på hvor lang tid man trenger å steke bacon i mikrobølgeovn for
gitte inputparametre.

