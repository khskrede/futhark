\chapter{Introduction}
%In this chapter, we don't need numbering of subsections (there are none).
\setcounter{secnumdepth}{1}

\section{Problem formulation}
The problem considered in this project is the preparation of bacon in a microwave oven. Bacon is
defined as cured meat from side and back cuts of a pig. A microwave oven in this context is a
household appliance typically delivering 750 W of microwave effect at a frequency of 2.1 GHz
to a Faraday cage of volume 40 liters. 

The motivation for the problem formulation was clear. As soon as modeling of food preparation was
brought on the table, bacon seemed a prime choice. A microwave oven may seem an odd choice for
bacon, but previous experiments have shown that optimal bacon can consistently be attained with this
preparation method. Obvious advantages over traditional bacon preparation include less cleaning up
to do afterwards and a shorter time-to-plate. The caveat is that microwave preparation is less
suitable for feedback during the cooking process, as the bacon is obscured from view. To the
traditional bacon chef, used to a touch-and-go approach, this is an obstacle to implementation, as
overcooking bacon in a microwave oven yields inedible results. The
purpose of this work, then, is to establish reliable numerical simulations that can serve as a basis
for estimating cooking time for an arbitrary slice of bacon.

As the project considers the optimal preparation of bacon in a microwave oven, two natural questions
were formulated:

\begin{itemize}
  \item How is preparation of bacon in a microwave oven modelled numerically?
  \item How does the size of the bacon affect the preparation, and what preparation time is optimal?
\end{itemize}

\section{Reasoning behind the problem formulation}
The first question is obvious, as any attempt at predicting the preparation time necessitates a
numerical model of bacon in a microwave oven; a simple glance at the relevant
equations tells the experienced numerics person that no analytical solution
exists. With a numerical solution of the heat and mass transport equations, the
optimal preparation time can be predicted. But this presumes an understanding of
what makes good bacon. So what does?

Work done by \cite{intarwebz} and \cite{maillard} suggest that two factors are important in
determining good bacon. The first is a high enough temperature that the Maillard
reaction can take place, which happens around 140$^{circ}$C. The second is that
a significant proportion of the fat has melted and has been transported out of
the bacon. Both of these factors are naturally affected by the composition of
the bacon; especially water and fat content will play a major role.

Having identified these two factors as the important criteria for an optimal
solution, experimental work is needed to determine the correlation between
temperature and fat loss, and how finished bacon is. This is, of course, also a
subjective question - different people prefer different grades of crispness.

Another experimental question is how much of the microwave effect that is
absorbed by the bacon. The second law of thermodynamics dictates that some of
the effect is lost in the microwave magnetron, the question is how much effect
is dissipated in the bacon. This may include losses besides that of the
magnetron, which has a typical efficiency of 65\% \cite{namba}. It turns out
that common household microwave ovens are rated to include this loss,
with standardized calibration procedures published by the IEC's SC 59K, so a 750 W
microwave oven will typically draw 1.1 kW of electrical power from the mains
socket. This means using the rated power will be good enough for our purposes.

Finally, experiments are needed to verify the accuracy of the numerical results.
As heat and mass transport is a Complicated Problem, this is not guaranteed a
priori.

%Reset secnumdepth for other chapters
\setcounter{secnumdepth}{2}
