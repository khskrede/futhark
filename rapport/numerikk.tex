\chapter{Numerikk}

\section{Von Neumann analysis of the Crank-Nicolson-scheme}

Consider the heat equation,
\begin{equation}
\frac{\partial u}{\partial t} = \alpha\frac{\partial^2u}{\partial^2x}
\label{eq:heat}
\end{equation}
By applying Crank-Nicolson the following iteration scheme is attained,
\begin{equation}
u_i^{n+1} = u_i^n\left(1-2D\right)+D\left(u_{i+1}^{n+1}-2u_i^{n+1}+u_{i-1}^{n+1}\right) + D\left(u_{i+1}^n + u_{i-1}^n\right)
\label{eq:crank}
\end{equation}
\begin{equation}
D = \frac{1}{2}\frac{\alpha\Delta t}{(\Delta x)^2}
\label{eq:crank-D}
\end{equation}
Using the approximation $u_j^n \approx E_j^n$ where $E_j^n = G^ne^{i\beta
x_j}$, and $n$ is a power, inserted into \cref{eq:heat}, this gives
\begin{equation*}
G^{n+1}e^{i\beta x_j} = G^ne^{i\beta x_j}\left(1-2D\right) + G^{n+1}D\left(e^{i\beta x_{j+1}} - 2e^{i\beta x_j} + e^{i\beta x_{j-1}}\right) + G^nD\left(e^{i\beta x_{j+1}} + e^{i\beta x_{j-1}}\right)
\end{equation*}
Dividing \cref{eq:crank-D} by $G^ne^{i\beta x_j}$, and using $x_{j+1} \approx
x_j + h$ one obtains
\begin{eqnarray*}
G\left[1-D\left(2\cos{\beta h} - 2\right)\right] &=& 1 - 2D\left(1-\cos{\beta h}\right) \\
\cos{\beta h} &=& \frac{1}{2}\left(1-\sin^2{\frac{\beta h}{2}}\right) \\
G\left(1+4D\sin^{2}{\frac{\beta h}{2}}\right) &=& 1 - 4D\sin^2{\frac{\beta h}{2}} \\
G &=& \frac{1-4D\sin^2{\frac{\beta h}{2}}}{1+4D\sin^2{\frac{\beta h}{2}}}
\end{eqnarray*}
As the maximum value of $\sin^2{x}$ is 1, and since the iteration scheme is
stable if $|G| \le 1$, this implies stability,
\begin{equation}
|G| = \left|\frac{1-4D}{1+4D}\right|
\end{equation}
%Marginpar makes a note in the margin, e.g. for explaining things.
\marginpar{A typical
\textbf{C}ourant-\textbf{F}riedrichs-\textbf{L}ewy condition is an inequality of the form
$\frac{\Delta t}{\Delta x} < C$, with C a constant.}
It is noteworthy that $D > 0 \ \Rightarrow \ |G| \le 1$ , so the Crank-Nicolson
scheme is \emph{unconditionally stable}. There is, however, a caveat: if the
time steps and space steps do not fulfill a CFL-condition, the scheme may
present oscillations.

\section{Truncation error in the Crank-Nicolson scheme}

The Crank-Nicholson method is based on the trapezoidal rule. From the trapezoidal rule the truncation error is represented by
\begin{equation*}
\int_0^k f(t) dt = \frac{1}{2} k (f(0) - f(k)) -\frac{1}{12} k^3 f''(\frac{k}{2}) + ...
\end{equation*}
Using the formula
\begin{equation*}
	u(x_m,t_{n+1}) - u(x_m,t_n) = \int_{t_n} ^{t_{n+1}} u_t(x_m,t) dt
\end{equation*}
and approximating the integral by the trapezoidal rule, and abbreviate the notation $u_m^{n+1/2} = u(x_m,t_n+\frac{1}{2} k)$, one obtain
\begin{eqnarray*}
u_m^ {n+1} &=& u_m^ n + \frac{1}{2} k(\partial_t u_m^n + \partial_t u_m^{n+1}) - \frac{1}{2} k^3 \partial_t ^3 u_m^{n+\frac{1}{2}} \\
		   &\stackrel{\text{\tiny heat eq.}}{=}&  u_m^ n + \frac{1}{2} k(\partial_x^2 u_m^n + \partial_y^2 u_m^n + \partial_z^2 u_m^n + \partial_x^2 u_m^{n+1} + \partial_y^2 u_m^{n+1} + \partial_z^2 u_m^{n+1}) - \frac{1}{2} k^3 \partial_t ^3 u_m^{n+\frac{1}{2}}
\end{eqnarray*}
Discretizing the double derivative with central differences, and denoting the step sizes in $x, y, z$ direction by $h, f, g$ respectively, gives
\begin{eqnarray*}
u_m^{n+1} &=&  u_m^ n + \frac{1}{2} k(\frac{1}{h^2}\delta_x^2 u_m^n + \frac{1}{f^2}\delta_y^2 u_m^n + \frac{1}{g^2}\delta_z^2 u_m^n + \frac{1}{h^2}\delta_x^2 u_m^{n+1} + \frac{1}{f^2}\delta_y^2 u_m^{n+1} + \frac{1}{g^2}\delta_z^2 u_m^{n+1}) \\
 		  &&- \frac{1}{2} k (\frac{1}{12}h^2\partial_x^4 u_m^n + \frac{1}{12}g^2\partial_y^4 u_m^n \frac{1}{12}f^2\partial_z^4 u_m^n + \frac{1}{12}h^2\partial_x^4 u_m^{n+1} + \frac{1}{12}g^2\partial_y^4 u_m^{n+1} \frac{1}{12}f^2\partial_z^4 u_m^{n+1} ) \\
 		  &&- \frac{1}{2} k^3 \partial_t ^3 u_m^{n+\frac{1}{2}} \\
 		  &=& u_m^n + \frac{r}{2}(\delta_x^2 u_m^n + \delta_x^2 u_m^{n+1}) + \frac{p}{2}(\delta_y^2 u_m^n + \delta_y^2 u_m^{n+1}) + \frac{q}{2}(\delta_z^2 u_m^n + \delta_z^2 u_m^{n+1}) + \tau_m^n
\end{eqnarray*}
where $r = k/h^2$, $p = k/f^2$, $q = k/g^2$, and $\tau_m^n$ is the truncation error times $k$. The expression for $\tau_m^n$ is
\begin{equation*}
\tau_m^n = -\frac{1}{12} k^3 \partial_t^2 u_m^{n+\frac{1}{2}} - \frac{1}{12} k h^2 \partial_x^4 u_m^{n+\frac{1}{2}} - \frac{1}{12} k g^2 \partial_y^4 u_m^{n+\frac{1}{2}} - \frac{1}{12} k f^2 \partial_z^4 u_m^{n+\frac{1}{2}}
\end{equation*}
where we have used $\frac{1}{2}(\partial_x^4 u_m^n + \partial_x^4 u_m^{n+}) = \partial_x^4 u_m^{n+\frac{1}{2}} + O(k^2)$. Hence, the truncation error is 
\begin{eqnarray*}
\frac{\tau_m^ n}{k} &=& -\frac{1}{12} k^2 \partial_t^2 u_m^{n+\frac{1}{2}} - \frac{1}{12} h^2 \partial_x^4 u_m^{n+\frac{1}{2}} - \frac{1}{12} g^2 \partial_y^4 u_m^{n+\frac{1}{2}} - \frac{1}{12} f^2 \partial_z^4 u_m^{n+\frac{1}{2}}\\
					&=& O(k^2 + h^2 + g^2 + f^2)
\end{eqnarray*}

