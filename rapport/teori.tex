\chapter{Theory}
\section{The physical system}
The physical system to be modeled is a thin, rectangular strip of bacon which is
subjected to approx. 750 watts of microwave radiation at the frequency of 2.4
GHz. To be modeled is the heat in the strip, phase transitions and transport.

\section{Our approach}
\begin{figure}[!h]
  \begin{center}
    \includegraphics[width=0.8\linewidth]{physicists.png}
  \end{center}
  \caption{Our initial approach to the problem}
  \label{fig:xkcd_physics}
\end{figure}

Initially, our approach to modelling was to use a staggered approach: first solve the
heat equation for the three media meat, solid fat, and liquid fat, thus giving
the temperature at step $n$. Then at step $n+1$, the temperature is regarded as
known, and we solve the transport equation for liquid fat out of the bacon. Then
we know the distribution of liquid fat at step $n+2$, where we solve the heat
equations again, and we repeat ad nauseam.

\section{Differential equations}
The differential equations governing the heat are all variations of the heat
equation, with various source terms, \cref{eq:temp}:
\begin{align}
  \label{eq:temp}
  (\rho c_p)_m \pd{T}{t} - \alpha_m \grad{^2 T} &= J^{MW} \quad \rm{m,} \\
  \eta_s (\rho c_p)_s \pd{T}{t} - \eta_s\alpha_s \grad{^2 T} &= J^{MW} - J^{Melt}  \quad \rm{s,} \\
  \eta_l (\rho c_p)_l \pd{T}{t} - \eta_s\alpha_l \grad{^2 T} - \eta_l(\rho c_p)_l(\v{v}\cdot
  \grad{})T &= J^{MW}  \quad \rm{l.}
\end{align}
Here the subscript $m$ denotes meat, $s$ denotes solid fat, and $l$ denotes
liquid fat. $J^{MW}$ is the source term representing the microwave oven.
In the initial approach, this is modelled as a cylindrically symmetric term,
with the radial power distribution on the form \cref{eq:effektfordeling},
\begin{equation}
  J^{MW}(r) = 0.5 + 2.55008x - 0.588013x^2 + 0.032445x^3 + 0.00124411x^4 - 0.0000973516x^5 \rm{,}
  \label{eq:effektfordeling}
\end{equation}
i.e. a fifth degree polynomial, interpolated from the results in \cite{huang+zhu}.

Furthermore, there is the term $J^{Melt}$ which represents heat loss due to
latent heat in the solid-liquid phase transition. This can be written as 
\[ \frac{L \rho}{T_2-T_1} \pd{T}{t} \rm{,} \quad T \in (T_1,T_2) \rm{,}\]
where the fat is melting between the temperatures $T_1$ and $T_2$. For fat these
are typically around 30-50$^\circ$C. For $T \notin (T1,T2)$ this term is zero,
and it is readily seen that this is equivalent to a modification of the constant
$c_{p,l}$ in \cref{eq:temp} when $T \in (T1,T2)$, thus it is not a serious
complication.

In the last of the heat equations, there also appears what is called a
convective derivative, on the form $(\v{v}\cdot \grad{})T$. This is a somewhat
more complicated term, and is essentially a modification due to transport of
liquid fat implying an implicit transport of heat.


