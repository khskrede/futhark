\chapter{Teori}
\section{Det fysiske systemet}
Det fysiske systemet som skal beskrives er en tynn, rektangulær plate med
spekket svinekjøtt som utsettes for mikrobølgestråling med en effekt på rundt
750 W. Det som skal modelleres er temperatur, faseoverganger for fett, og
transport av fett.

\section{Differensiallikninger}
De differensiallikningene som beskriver det fysiske systemet kan deles inn i to
klasser: varmelikningen med kildeledd, og Navier-Stokes likning med kildeledd.
Svinekjøttet deles inn i tre materialer: kjøtt, fast fett og flytende fett. Hver
av disse materialene vil ha en egen varmelikning med ulike ledd, \cref{eq:temp}:
\begin{align}
  \label{eq:temp}
  \pd{T}{t} + \grad{^2 T} &= J^{MW} \quad \rm{m,} \\
  \pd{T}{t} + \grad{^2 T} &= J^{MW}  \quad \rm{s,} \\
  \pd{T}{t} + \grad{^2 T} &= J^{MW}  \quad \rm{l.}
\end{align}
Her står m for kjøtt (meat), s for fast fett, og l for flytende fett. $J^{MW}$
er mikrobølgeovnene, representert som et kildeledd. Dette modelleres i første
omgang som et sylindersymmetrisk effektivt ledd, med en radiell form gitt ved
\cref{eq:effektfordeling},
\begin{equation}
  J^{MW}(r) = -1.33789 + 4.54712x - 1.25814x^2 + 0.127253x^3 - 0.00469462x^4 + 0.0000385689x^5 \rm{,}
  \label{eq:effektfordeling}
\end{equation}
altså et 5. gradspolynom. Dette er tilpasset numeriske resultater fra \cite{huang+zhu}
