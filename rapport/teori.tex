\chapter{Teori}
\section{Det fysiske systemet}
Det fysiske systemet som skal beskrives er en tynn, rektangulær plate med
spekket svinekjøtt som utsettes for mikrobølgestråling med en effekt på rundt
750 W. Det som skal modelleres er temperatur, faseoverganger for fett, og
transport av fett.

\section{Differensiallikninger}
De differensiallikningene som beskriver det fysiske systemet kan deles inn i to
klasser: varmelikningen med kildeledd, og Navier-Stokes likning med kildeledd.
Svinekjøttet deles inn i tre materialer: kjøtt, fast fett og flytende fett. Hver
av disse materialene vil ha en egen varmelikning med ulike ledd, \cref{eq:temp}:
\begin{align}
  \label{eq:temp}
  (\rho c_p)_m \pd{T}{t} - \alpha_m \grad{^2 T} &= J^{MW} \quad \rm{m,} \\
  \eta_s (\rho c_p)_s \pd{T}{t} - \eta_s\alpha_s \grad{^2 T} &= J^{MW} - J^{Smelting}  \quad \rm{s,} \\
  \eta_l (\rho c_p)_l \pd{T}{t} - \eta_s\alpha_l \grad{^2 T} - \eta_l(\rho c_p)_l(\v{v}\cdot
  \grad{})T &= J^{MW}  \quad \rm{l.}
\end{align}
Her står m for kjøtt (meat), s for fast fett, og l for flytende fett. $J^{MW}$
er mikrobølgeovnene, representert som et kildeledd. Dette modelleres i første
omgang som et sylindersymmetrisk effektivt ledd, med en radiell form gitt ved
\cref{eq:effektfordeling},
\begin{equation}
  J^{MW}(r) = 0.5 + 2.55008x - 0.588013x^2 + 0.032445x^3 + 0.00124411x^4 - 0.0000973516x^5 \rm{,}
  \label{eq:effektfordeling}
\end{equation}
altså et 5. gradspolynom. Dette er fra numeriske resultater fra \cite{huang+zhu}.

Videre har vi leddene $J^{Smelting}$ som er varmetapet på grunn av smelting av
fast fett. Dette kan skrives på formen \[ \frac{L \rho}{T_2-T_1} \pd{T}{t}
\rm{,} \quad T \in (T1,T2) \rm{,}\] der fettet smelter mellom temperaturene T1
og T2. For $T \notin (T1,T2)$ er dette leddet null. Vi ser da at dette leddet kun
blir en modifikasjon av konstanten foran $\pd{T}{t}$-leddet, og dermed ikke en
stor komplikasjon.

I den siste likningen har man også en konvektiv derivert, på formen $(\v{v}\cdot
\grad{})T$. Dette representerer varme som transporteres 

