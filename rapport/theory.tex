\chapter{Theory}
\section{The physical system}
The physical system to be modeled is a thin, rectangular strip of bacon which is
subjected to approx. 750 watts of microwave radiation at the frequency of 2.4
GHz. To be modeled is the heat in the strip, phase transitions and transport.

\section{Our approach}
\begin{figure}[!h]
  \begin{center}
    \includegraphics[width=0.4\linewidth]{physicists.png}
  \end{center}
  \caption{Our initial approach to the problem. \url{xkcd.com/793}}
  \label{fig:xkcd_physics}
\end{figure}

Initially, our approach to modelling was to use a staggered approach: first solve the
heat equation for the three media meat, solid fat, and liquid fat, thus giving
the temperature at step $n$. Then at step $n+1$, the temperature is regarded as
known, and we solve the transport equation for liquid fat out of the bacon. Then
we know the distribution of liquid fat at step $n+2$, where we solve the heat
equations again, and we repeat ad nauseam.

\section{The heat equations}
The differential equations governing the heat transport are all variations of the heat
equation, with various source terms, \cref{eq:temp}:
\begin{align}
  \label{eq:temp}
  (\rho c_p)_m \pd{T}{t} - \alpha_m \grad{^2 T} &= J^{MW} \quad \rm{m,} \\
  \eta_s (\rho c_p)_s \pd{T}{t} - \eta_s\alpha_s \grad{^2 T} &= J^{MW} - J^{Melt}  \quad \rm{s,} \\
  \eta_l (\rho c_p)_l \pd{T}{t} - \eta_s\alpha_l \grad{^2 T} - \eta_l(\rho c_p)_l(\v{v}\cdot
  \grad{})T &= J^{MW}  \quad \rm{l.}
\end{align}
Here the subscript $m$ denotes meat, $s$ denotes solid fat, and $l$ denotes
liquid fat. $J^{MW}$ is the source term representing the microwave oven.
In the initial approach, this is modelled as a cylindrically symmetric term,
with the radial power distribution on the form \cref{eq:effektfordeling},
\begin{equation}
  J^{MW}(r) = 0.5 + 2.55008x - 0.588013x^2 + 0.032445x^3 + 0.00124411x^4 - 0.0000973516x^5 \rm{,}
  \label{eq:effektfordeling}
\end{equation}
i.e. a fifth degree polynomial, interpolated from the results in \cite{huang+zhu}.

Furthermore, there is the term $J^{Melt}$ which represents heat loss due to
latent heat absorbed in the solid-liquid phase transition. This can be written as 
\[ \frac{L \rho}{T_2-T_1} \pd{T}{t} \rm{,} \quad T \in (T_1,T_2) \rm{,}\]
where the fat is melting between the temperatures $T_1$ and $T_2$. For fat these
are typically around 30-50$^\circ$C. For $T \notin (T1,T2)$ this term is zero,
and it is readily seen that this is equivalent to a modification of the constant
$c_{p,l}$ in \cref{eq:temp} when $T \in (T1,T2)$, thus it is not a serious
complication.

In the last of the heat equations, there also appears what is called a
convective derivative, on the form $(\v{v}\cdot \grad{})T$. This is a somewhat
more complicated term, and is essentially a modification due to transport of
liquid fat implying an implicit transport of heat. But, we're in luck, boys! 

As the bacon strip is essentially two-dimensional, transport will almost
exclusively happen in the vertical (small) direction. In our approach, the
source terms don't vary in the vertical direction, so this term will be a transport
of heat in the vertical direction. This will not affect the heat transport in
the horisontal directions.  As we're not really
interested in the vertical temperature distribution, this means that we can
happily neglect this term, in the spirit of Richard Feynmann: ``it doesn't give
any new physics''.


\section{The transport equations}
The transport equation used in our approach is the one-dimensional Navier Stokes
equation, with the additional assumptions of an incompressible, Newtonian flow.
This is the same as ignoring phenomena where sound waves or shock waves are
important, which makes sense here. This leads to \cref{eq:navier-stokes}
\begin{equation}
  \rho \left(\pd{v}{t} + v \cdot \pd{v}{x}\right) = -\pd{p}{x} + \mu \nabla^2 v + f
 \label{eq:navier-stokes}
\end{equation}
where $f$ includes any additional forces, such as gravity. This term is
discussed in further detail below. An interesting term here is $\pd{p}{x}$, where
$p$ is the pressure in the fluid. Following the work by \cite{brent}, we replace
this with a term inspired by \cref{carman-kozeny}
\begin{equation}
  \pd{p}{x} = -C\frac{ (1-\eta)^2}{\eta^3} v_a
  \label{carman-kozeny}
\end{equation}
where $\eta$ is the melted fraction in the current volume element, and
$v_a$ is an apparent velocity. As the volume element melts, $\eta$ goes from
zero to one, and $\pd{p}{x}$ goes from a very large value to zero. This is known 
as the Carman-Kozeny equation for flow in a pouros medium, and is applicable to 
materials where melting does not happen at one specific temperature, but instead 
over a range of temperatures (non-isothermal phase change). \cite{poirier} 

This is typical for a material with a combination of slightly different
fats, e.g. chocolate or in this case bacon. When a volume element is partially
melted, it is assumed to have a dendritic structure, and it is said to be in a
``mushy'' phase. The dendritic structure itself, and whether the flow is parallell or
orthogonal to the dendritic structure, will influence the constant C. 

Using the procedure outlined in \cite{poirier}, the constant $C$ was estimated to be
$8 \times 10^8$, where flow is assumed to be parallell to the dendritic
structure, and the primary dendritic arm distance in bacon was estimated to be
2.5 mm. The assumption of parallell flow dictates the value of the Kozeny
constant yielding the final value for $C$.

Influenced by these considerations, the term $\pd{p}{x}$ in
\cref{eq:navier-stokes} was replaced by \cref{eq:pressure}
\begin{equation}
  \pd{p}{x} =  -C\frac{ (1-\eta)^2}{(\eta+d)^3} v
  \label{eq:pressure}
\end{equation}
where $d$ is a tiny constant introduced simply to avoid dividing by zero in the
numerical procedure.

Introducing this term has two effects. First of all, for a solid element,
$\pd{p}{x}$ becomes very large, and forces the velocity to be zero. When the
element starts melting, the flow will mimic the Carman-Kozeny flow, which is
realistic for a porous medium, and finally in the molten stage, the ordinary
Navier Stokes equations are recovered.



