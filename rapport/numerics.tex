\chapter{Numerics}

\section{Numerical schemes}

In this project we have use the Cranch-Nicholson method to discretize the heat
equation \cref{eq:temp} in 3D over a finite grid. The Crank-Nicholson method is based on the equation \cref{integral}, where the trapezoidal rule \cref{trapezoidalrule} is used to approximate the integral on the right hand side. We insert the right hand side of the heat equation in the resulting expression \cref{res}, and use Taylor-expansion to obtain the discretizaion. We use the notation \cref{notations} in the rest of the document.
\begin{equation}
\partial_{x}^{k}u=\frac{\partial^{k}u}{\partial^{}x^{k}},\quad \quad \partial_{t}^{k}u=\frac{\partial^{k}u}{\partial^{}t^{k}} \quad \quad u_{m}^{n+\alpha k } = u(x_{m},t_{n}+\alpha k)
\label{notations}
\end{equation}

\begin{equation}
u(x_m,t_{n+1}) - u(x_m,t_n) = \int_{t_n} ^{t_{n+1}} u_t(x_m,t) dt
\label{integral}
\end{equation}

\begin{equation}
\int_0^k f(t) dt = \frac{1}{2} k (f(0) - f(k)) -\frac{1}{12} k^3 f''(\frac{k}{2}) + ...
\label{trapezoidalrule}
\end{equation}
We thus obtain
\begin{eqnarray}
\label{res}
u_m^ {n+1} &=& u_m^ n + \frac{1}{2} k(\partial_t u_m^n + \partial_t u_m^{n+1}) - \frac{1}{12} k^3 \partial_t ^3 u_m^{n+\frac{1}{2}} \\
&\stackrel{\text{\tiny heat eq.}}{=}& u_m^ n + \frac{1}{2} k(\partial_x^2 u_m^n + \partial_y^2 u_m^n + \partial_z^2 u_m^n + \partial_x^2 u_m^{n+1} + \partial_y^2 u_m^{n+1} + \partial_z^2 u_m^{n+1}) - \frac{1}{12} k^3 \partial_t ^3    u_m^{n+\frac{1}{2}} \nonumber
\end{eqnarray}
Using central differences \cref{centraldiff}, and denoting the step sizes in $x, y, z$ direction by $h, f, g$ respectively, gives
\begin{eqnarray*}
u_m^{n+1} &=& u_m^ n + \frac{1}{2} k(\frac{1}{h^2}\delta_x^2 u_m^n + \frac{1}{f^2}\delta_y^2 u_m^n + \frac{1}{g^2}\delta_z^2 u_m^n + \frac{1}{h^2}\delta_x^2 u_m^{n+1} + \frac{1}{f^2}\delta_y^2 u_m^{n+1} + \frac{1}{g^2}\delta_z^2 u_m^{n+1}) \\
  &&- \frac{1}{2} k (\frac{1}{12}h^2\partial_x^4 u_m^n + \frac{1}{12}g^2\partial_y^4 u_m^n + \frac{1}{12}f^2\partial_z^4 u_m^n + \frac{1}{12}h^2\partial_x^4 u_m^{n+1} + \frac{1}{12}g^2\partial_y^4 u_m^{n+1} + \frac{1}{12}f^2\partial_z^4 u_m^{n+1} + ...) \\
  &&- \frac{1}{12} k^3 \partial_t ^3 u_m^{n+\frac{1}{2}} \\ 
  %&=& u_m^n + \frac{r}{2}(\delta_x^2 u_m^n + \delta_x^2 u_m^{n+1}) + \frac{p}{2}(\delta_y^2 u_m^n + \delta_y^2 u_m^{n+1}) + \frac{q}{2}(\delta_z^2 u_m^n + \delta_z^2 u_m^{n+1}) + \tau_m^n
\end{eqnarray*}
where
\begin{equation}
\delta_{r}^{2}U_m^{n}=\frac{U_{m+1}^{n}-2U_m^{n}+U_{m-1}^{n}}{{\Delta r}^{2}}
\label{centraldiff}
\end{equation}
We thus obtain the implicit method for the heat equation
\begin{equation}
u_m^{n+1}-k(\frac{1}{h^2}\delta_x^2 u_m^{n+1}-\frac{1}{f^2}\delta_y^2 u_m^{n+1}-\frac{1}{g^2}\delta_z^2 u_m^{n+1})=u_m^n + k(\frac{1}{h^2}\delta_x^2 u_m^n + \frac{1}{f^2}\delta_y^2 u_m^n +\frac{1}{g^2}\delta_z^2 u_m^n)
\label{crank}
\end{equation}
with truncation error error
\begin{eqnarray}
\frac{\tau_m^ n}{k} &=& -\frac{1}{12} k^2 \partial_t^2 u_m^{n+\frac{1}{2}} - \frac{1}{12} h^2 \partial_x^4 u_m^{n+\frac{1}{2}} - \frac{1}{12} g^2 \partial_y^4 u_m^{n+\frac{1}{2}} - \frac{1}{12} f^2 \partial_z^4 u_m^{n+\frac{1}{2}}\\
&=& O(k^2 + h^2 + g^2 + f^2)
\label{truncerror}
\end{eqnarray}

Here we have used that $\frac{1}{2}(\partial_x^4 u_m^n + \partial_x^4 u_m^{n+1}) = \partial_x^4 u_m^{n+\frac{1}{2}} + O(k^2)$. 
The truncation error $\tau_m^n \Rightarrow 0$ as $h,f,g,k \Rightarrow 0$, and the Crank-Nicholson method is consistent for the heat equation. To see if the method converges we perform a von Neumann analysis of the numerical scheme.

\section{Von Neumann analysis of the Crank-Nicholson-scheme}

The von Neumann analysis is based on Fourier analysis. The method consist of substituting 
\begin{equation*}
	U_m^n=\xi^n e^{i \beta x_m} \quad \quad  i=\sqrt{-1}
\end{equation*}
in the difference equation and solve for $\xi$.
For the method to be stable it has to meet the condition
\begin{equation}
	\mid{\xi}\mid \leq 1
	\label{stabcond}
\end{equation}

Here we only perform the Neumann analysis for the one-dimensional heat equation problem. We thus obtain the expression
\begin{equation*}
\xi^{n+1} e^{i\beta x_{j}} = \xi^{n} e^{i\beta x_j}\left(1-2D\right) + \xi^{n+1}D\left(e^{i\beta x_{j+1}} - 2e^{i\beta  x} + e^{i\beta x_{j-1}}\right) + \xi^{nD}\left(e^{i\beta x_{j+1}} + e^{i\beta x_{j-1}}\right)
\end{equation*}

where
\begin{equation*}
D = \frac{1}{2}\frac{\alpha\Delta t}{(\Delta x)^2}
\label{eq:crank-D}
\end{equation*}
\marginpar{A typical
\textbf{C}ourant-\textbf{F}riedrichs-\textbf{L}ewy condition is an inequality of the form
$\frac{\Delta t}{\Delta x} < C$, with C a constant. }
%Marginpar makes a note in the margin, e.g. for explaining things.

Dividing  by $\xi^ne^{i\beta x_j}$, and using $x_{j+1} = x_j + h$ one obtains    %\href{eq:crank-D}
\begin{eqnarray*}
\xi\left[1-D\left(2\cos{\beta h} - 2\right)\right] &=& 1 - 2D\left(1-\cos{\beta h}\right) \\
\cos{\beta h} &=& 1-2\sin^2{\frac{\beta h}{2}} \\
\xi\left(1+4D\sin^{2}{\frac{\beta h}{2}}\right) &=& 1 - 4D\sin^2{\frac{\beta h}{2}} \\
\xi &=& \frac{1-4D\sin^2{\frac{\beta h}{2}}}{1+4D\sin^2{\frac{\beta h}{2}}}
\end{eqnarray*}
As the maximum value of $\sin^2{x}$ is 1, the methods meets the stability condition \ref{stabcond}.
\begin{equation}
|\xi | = \left|\frac{1-4D}{1+4D}\right|
\end{equation}

Since D always are positiv, we get that the Crank-Nicolson
scheme for the heat equation is \emph{unconditionally stable}. In practice it may happen that the method oscillates if the time steps and space steps do not fulfill a CFL-condition.
\\
\\
From Lax' equivalent theorem, which state that a consistent difference scheme will converges if and only if it is stable, the Crank-Nicholson scheme will converge.


\section{Truncation error in the Crank-Nicolson scheme}

The Crank-Nicholson method is based on the trapezoidal rule. From the trapezoidal rule the truncation error is represented by
\begin{equation*}
\int_0^k f(t) dt = \frac{1}{2} k (f(0) - f(k)) -\frac{1}{12} k^3 f''(\frac{k}{2}) + ...
\end{equation*}
Using the formula
\begin{equation*}
	u(x_m,t_{n+1}) - u(x_m,t_n) = \int_{t_n} ^{t_{n+1}} u_t(x_m,t) dt
\end{equation*}
and approximating the integral by the trapezoidal rule, and abbreviate the notation $u_m^{n+1/2} = u(x_m,t_n+\frac{1}{2} k)$, one obtain
\begin{eqnarray*}
u_m^ {n+1} &=& u_m^ n + \frac{1}{2} k(\partial_t u_m^n + \partial_t u_m^{n+1}) - \frac{1}{2} k^3 \partial_t ^3 u_m^{n+\frac{1}{2}} \\
		   &\stackrel{\text{\tiny heat eq.}}{=}&  u_m^ n + \frac{1}{2} k(\partial_x^2 u_m^n + \partial_y^2 u_m^n + \partial_z^2 u_m^n + \partial_x^2 u_m^{n+1} + \partial_y^2 u_m^{n+1} + \partial_z^2 u_m^{n+1}) - \frac{1}{2} k^3 \partial_t ^3 u_m^{n+\frac{1}{2}}
\end{eqnarray*}
Discretizing the double derivative with central differences, and denoting the step sizes in $x, y, z$ direction by $h, f, g$ respectively, gives
\begin{eqnarray*}
u_m^{n+1} &=&  u_m^ n + \frac{1}{2} k(\frac{1}{h^2}\delta_x^2 u_m^n + \frac{1}{f^2}\delta_y^2 u_m^n + \frac{1}{g^2}\delta_z^2 u_m^n + \frac{1}{h^2}\delta_x^2 u_m^{n+1} + \frac{1}{f^2}\delta_y^2 u_m^{n+1} + \frac{1}{g^2}\delta_z^2 u_m^{n+1}) \\
 		  &&- \frac{1}{2} k (\frac{1}{12}h^2\partial_x^4 u_m^n + \frac{1}{12}g^2\partial_y^4 u_m^n \frac{1}{12}f^2\partial_z^4 u_m^n + \frac{1}{12}h^2\partial_x^4 u_m^{n+1} + \frac{1}{12}g^2\partial_y^4 u_m^{n+1} \frac{1}{12}f^2\partial_z^4 u_m^{n+1} ) \\
 		  &&- \frac{1}{2} k^3 \partial_t ^3 u_m^{n+\frac{1}{2}} \\
 		  &=& u_m^n + \frac{r}{2}(\delta_x^2 u_m^n + \delta_x^2 u_m^{n+1}) + \frac{p}{2}(\delta_y^2 u_m^n + \delta_y^2 u_m^{n+1}) + \frac{q}{2}(\delta_z^2 u_m^n + \delta_z^2 u_m^{n+1}) + \tau_m^n
\end{eqnarray*}
where $r = k/h^2$, $p = k/f^2$, $q = k/g^2$, and $\tau_m^n$ is the truncation error times $k$. The expression for $\tau_m^n$ is
\begin{equation*}
\tau_m^n = -\frac{1}{12} k^3 \partial_t^2 u_m^{n+\frac{1}{2}} - \frac{1}{12} k h^2 \partial_x^4 u_m^{n+\frac{1}{2}} - \frac{1}{12} k g^2 \partial_y^4 u_m^{n+\frac{1}{2}} - \frac{1}{12} k f^2 \partial_z^4 u_m^{n+\frac{1}{2}}
\end{equation*}
where we have used $\frac{1}{2}(\partial_x^4 u_m^n + \partial_x^4 u_m^{n+}) = \partial_x^4 u_m^{n+\frac{1}{2}} + O(k^2)$. Hence, the truncation error is 
\begin{eqnarray*}
\frac{\tau_m^ n}{k} &=& -\frac{1}{12} k^2 \partial_t^2 u_m^{n+\frac{1}{2}} - \frac{1}{12} h^2 \partial_x^4 u_m^{n+\frac{1}{2}} - \frac{1}{12} g^2 \partial_y^4 u_m^{n+\frac{1}{2}} - \frac{1}{12} f^2 \partial_z^4 u_m^{n+\frac{1}{2}}\\
					&=& O(k^2 + h^2 + g^2 + f^2)
\end{eqnarray*}

