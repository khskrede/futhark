\chapter{Implementation}

\section{Tools}

The programming language chosen to implemented the system was c++.
The arguments leading to this choice was:

\begin{itemize}
\item Most of the group members had some earlier experience with the language.
\item It allows for a high level of abstraction without loosing control of the
underlying hardware.
\item The code can be easily parallelized for shared memory machines using OpenMP.
\item Many libraries exist for c++ that implements highly optimized linear algebra
functionality.
\end{itemize}

Other languages cosidered where Matlab and c.

The tool chosen to visualize the resulting data was gnuplot and ffmpeg. Writing a
program using OpenGL was considered, but dismissed on the grounds of beeing too time consuming.

\section{Program design}

The program mainly consist of three parts:

\begin{itemize}
\item C++ implementation of Linear algebra functionality of the discrete equations.
\item C++ implementation of the conjugate gradient method for solving the
equations described.
\item Bash script for generating plots and and movie of the generated data.
\end{itemize}

The Bash script is not discussed in further detail as it can be considered simple.

\section{The physical system}

The main task of the physical system is to implement a matrix-vector multiplication
procedure for the heat equation and a method of calculating the heating properties of
a microwave.

\subsection{Partitioning of the problem}

\subsection{Updating alpha and beta values}

\subsection{Calculating the distribution of the microwave effect}

\subsection{The matrix-vector multiplication procedure}

\section{The conjugate gradient method as an iterative algorithm}


